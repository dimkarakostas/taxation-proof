\section{Incentivizing User Participation}\label{sec:incentives}

In Section~\ref{sec:provisions-extension} we presented a mechanism that allows
the taxation authority to identify a tax evasion by an exchange. A core
assumption of our scheme is that the exchange's users actively participate in
the address verification protocol $\taxationAddressProto$ and, in case of
discrepancies from the exchange, formally complain to the taxation authority.
However, it is possible that users do not care about the taxation of the
exchange or don't want to spend the time and effort required by
$\taxationAddressProto$. Therefore, it is important to consider incentives,
such that users participate in the protocol and are rewarded for their effort.

\subsection{Desiderata of the Incentive Mechanism}\label{subsec:incentive-desiderata}

The incentive mechanism should offer the following properties:

\begin{itemize}
    \item \textbf{Minimum trust to authority}: the mechanism should assume a
        minimum level of trust to and need for intervention from the taxation
        authority $\taxAuth$
    \item \textbf{Small deposit}: the mechanism should not require a large
        amount of assets to be deposited and locked by the exchange
    \item \textbf{Automated rewards}: as user should be automatically
        rewarded if they \emph{provably} identify a tax evasion
    \item \textbf{Integrity}: every user who can prove a tax evasion should be
        rewarded
    \item \textbf{Validity}: if a user forges a tax evasion statement they
        should not be rewarded (equiv. the exchange should not be penalized
        unless it has participated in tax evasion)
\end{itemize}

\subsection{The Reward Smart Contract}\label{subsec:reward-contract}

The reward smart contract for the user verification is defined in
Figure~\ref{fig:reward_contract}.

\begin{figure}[h]
\begin{mdframed}

\begin{center}
    \textbf{Smart Contract} $\rewardContract$
\end{center}

    % \vspace{0.3cm}
    $\rewardContract$ interacts with a set of parties $\partyset$, the exchange
    $\exchange$, and the taxation party $\taxAuth$, and holds the following:
    \begin{itemize}[$\cdot$]
        \item $\reward$: the reward, \ie a deposit in cryptocurrency;
        \item $\verifykey_\exchange$: the public key of the exchange;
        \item $L$: a list of commitments to addresses;
        \item $Y$: the list of public keys used for anonymity;
        \item $v$: the dispute flag, initially set to $\bot$.
    \end{itemize}

    \begin{itemize}
        \item If $v = \bot$:
            \begin{itemize}
                \item Upon receiving a message
                    $\msg{CheckTaxEvasion}{\transcript, \sig}$ from a party
                    $\party \in \partyset$, where $\transcript = \langle i,
                    \theta, \lambda, c \rangle$, check:
                    \begin{enumerate}
                        \item $\theta \stackrel{?}{=} \lambda \cdot (L[i] \cdot Y[i]^{-1})^c$
                        \item $\algoverify(\transcript, \sig, \verifykey_\exchange) \stackrel{?}{=} 1$
                    \end{enumerate}
                    If either fails then set $v := \party$.
                \item Upon receiving $\msg{UpdateAddr}{L'}$ from $\exchange$,
                    set $L := L'$.
                \item Upon receiving $\msg{UpdateComm}{Y'}$ from $\exchange$,
                    set $Y := Y'$.
            \end{itemize}

        \item If $v \neq \bot$, upon receiving a message $\msg{ResolveDispute}{f}$ from
            $\taxAuth$, where $f \in \{0, 1\}$, if $f = 1$ then transfer
            $\reward$ to $v$, else set $v := \bot$.

        \item Upon receiving $\msg{PayReward}{\party}$ from $\taxAuth$,
            transfer $r$ to $\party$.
    \end{itemize}

\end{mdframed}
\caption{
    The Taxation Reward smart contract, which incentivizes users to complete the
    address verification protocol $\taxationAddressProto$ and ensure that the
    exchange properly registers its assets for taxation purposes.
}
\label{fig:reward_contract}
\end{figure}
