\section{Incentivizing User Participation}\label{sec:incentives}

In Section~\ref{sec:provisions-extension} we presented a mechanism that allows
the taxation authority to identify a tax evasion by an exchange. A core
assumption of our scheme is that the exchange's users actively participate in
the address verification protocol $\taxationAddressProto$ and, in case of
discrepancies from the exchange, formally complain to the taxation authority.
However, it is possible that users do not care about the taxation of the
exchange or don't want to spend the time and effort required by
$\taxationAddressProto$. Therefore, it is important to consider incentives,
such that users participate in the protocol and are rewarded for their effort.

\subsection{Desiderata of the Incentive Mechanism}\label{subsec:incentive-desiderata}

The incentive mechanism should offer the following properties:

\begin{itemize}
    \item \textbf{Minimum trust to authority}: the mechanism should assume a
        minimum level of trust to and need for intervention from the taxation
        authority $\taxAuth$
    \item \textbf{Small deposit}: the mechanism should not require a large
        amount of assets to be deposited and locked by the exchange
    \item \textbf{Automated rewards}: as user should be automatically
        rewarded if they \emph{provably} identify a tax evasion
    \item \textbf{Integrity}: every user who can prove a tax evasion should be
        rewarded
    \item \textbf{Validity}: a user should should not be rewarded for a false
        tax evasion complaint (equiv. the exchange should not be penalized
        unless it has participated in tax evasion)
\end{itemize}

Challenge:
    The exchange might refuse to run protocol.
Argument:
    The exchange performs the transaction and \emph{after} that the user
    requests for taxation verification. The exchange cannot know a priori which
    user will ask for proof and does not know if the user works with/for the
    taxation authority.

\subsection{Automated Reward Impossibility}

Conjecture:
    If the verifier (\ie the user) is not honest, the protocol
    $\taxationAddressProto$ is not sufficient to construct an automated reward
    mechanism that satisfies validity.

Argument:
    $\taxationAddressProto$ is interactive, so its transcript is not sufficient
    to prove that the user is not lying (specifically, the user can pick $s, c$
    at random and set $\lambda = g^\theta y^{-c}$) \comment{would a NIZK suffice?}

Proposition:
    The exchange supplies the user with $t_i$.

Problems:
    $y_i$ at some point, \ie the exchange undeniably deanonymizes the address
    $y_i$.

Proposition:
    Assume a contract $S$ which solves the disputes. The user challenges the
    exchange for some address $y_i$. The exchange

\subsection{The Reward Smart Contract}\label{subsec:reward-contract}

We now present a smart contract definition that covers the desiderata defined
above. The core idea is that the smart contract will verify the transcript of
$\taxationAddressProto$ which was performed between the user and the exchange
and, in case the exchange did not behave properly, will reward the user who
submits the request.

First, the contract receives a deposit $\reward$. For instance, if the smart
contract is deployed on Ethereum, the deposit takes the form of $\reward$
Ether. The deposit should be large enough, in order to incentivize a user to
follow the protocol, but also small enough such that the overall mechanism is
cost-effective for the exchange. The exact deposit amount depends on the
application details and the game theoretic assumptions on the user behavior,
which are outside the scope of this paper and will be explored in future work.

In order for the \emph{validity} desideratum to hold, it is necessary that the
contract is able to identify whether the exchange participated in the protocol
execution defined in the transcript. Therefore, we need a mechanism to
automatically detect whether a user submits a forged transcript, which does not
correspond to a correct interaction with the exchange. We solve this problem by
requiring that the exchange signs the transcript of $\taxationAddressProto$ and
provides the signature to the user. Therefore, the contract is parameterized,
upon instantiation, by the public key of the exchange, in order to verify the
signatures submitted by the user.

The reward smart contract for the user verification of the exchange's addresses
is defined in Figure~\ref{fig:reward_contract}.

\begin{figure}[h]
\begin{mdframed}

\begin{center}
    \textbf{Smart Contract} $\rewardContract$
\end{center}

    % \vspace{0.3cm}
    $\rewardContract$ interacts with a set of parties $\partyset$, the exchange
    $\exchange$, and the taxation party $\taxAuth$, and holds the following:
    \begin{itemize}
        \item $\reward$: the reward, \ie a deposit in cryptocurrency;
        \item $\verifykey_\exchange$: the public key of the exchange;
        \item $L$: a list of commitments to addresses;
        \item $Y$: the list of public keys used for anonymity;
        \item $v$: the dispute flag, initially set to $\bot$.
    \end{itemize}

    \begin{itemize}
        \item If $v = \bot$:
            \begin{itemize}
                \item Upon receiving a message
                    $\msg{CheckTaxEvasion}{\transcript, \sig}$ from a party
                    $\party \in \partyset$, where $\transcript = \langle i,
                    \theta, \lambda, c \rangle$, check:
                    \begin{enumerate}
                        \item $\theta \stackrel{?}{=} \lambda \cdot (L[i] \cdot Y[i]^{-1})^c$
                        \item $\algoverify(\transcript, \sig, \verifykey_\exchange) \stackrel{?}{=} 1$
                    \end{enumerate}
                    If either fails then set $v := \party$.
                \item Upon receiving $\msg{UpdateAddr}{L'}$ from $\exchange$,
                    set $L := L'$.
                \item Upon receiving $\msg{UpdateComm}{Y'}$ from $\exchange$,
                    set $Y := Y'$.
            \end{itemize}

        \item If $v \neq \bot$, upon receiving a message $\msg{ResolveDispute}{f}$ from
            $\taxAuth$, where $f \in \{0, 1\}$, if $f = 1$ then transfer
            $\reward$ to $v$, else set $v := \bot$.

        \item Upon receiving $\msg{PayReward}{\party}$ from $\taxAuth$,
            transfer $r$ to $\party$.
    \end{itemize}

\end{mdframed}
\caption{
    The Taxation Reward smart contract, which incentivizes users to complete the
    address verification protocol $\taxationAddressProto$ and ensure that the
    exchange properly registers its assets for taxation purposes.
}
\label{fig:reward_contract}
\end{figure}
