\section{Properties and Limitations of Cryptocurrency Taxation}\label{sec:taxation}

Our goal is to design a mechanism which prevents tax evasion while retaining as
much privacy as possible. A basic requirement is that the user $\user$ declares
its assets to the taxation authority $\taxAuth$.  Specifically, in every
cryptocurrency system, a user $\user$ manages their assets via
\emph{addresses}. Each address $\address$ is associated with a key pair
$\keypair$, such that the private key $\signkey$ is used to claim ownership of
the assets, \eg by signing special messages; typically $\address =
\mathsf{H}(\verifykey)$ for some hash function $\mathsf{H}$. Assuming
$[\address_1, \dots, \address_n]$ is the list of all addresses that $\user$
controls, the total amount of assets that $\user$ owns is $\assets =
\sum_{i=1}^n \balance(\address_i)$, where $\balance(\address)$ is the (publicly
available) amount of assets controlled by the address $\address$.  Ideally,
$\assets$ is the only information that is leaked to $\taxAuth$ by the taxation
mechanism, while ensuring that the user cannot hide any assets.

\paragraph{Limitations.}\label{subsec:limitations}

We showcase the limitations of current systems with a telling example.  Assume
that Alice tax evades, \ie creates a secret, undeclared address $\address$ and
controls some assets $\theta$ in it. Given the pseudonymity (or anonymity) of blockchain systems, as long as $\address$ is kept hidden, \ie
Alice does not use it, it is impossible for anyone to identify or prove that
Alice owns $\address$ and, consequently, $\theta$.
Now assume that, in the future, Alice issues a transaction $\tau$ which
sends $\theta$ assets from $\address$ to Bob. If Bob suspects that Alice evaded
taxation for these $\theta$ assets, they might want to report them to the
taxation authority $\taxAuth$ for inspection, \ie they would form a complaint
that Alice owns $\address$ and evades taxation for it.
When $\taxAuth$ receives Bob's complaint, it faces a problem. Even if
$\taxAuth$ has proof that Alice did not declare $\address$ for taxation, it
further needs proof that $\address$ is controlled by Alice, \ie a proof
that Alice knows the private key associated with $\address$. Without such
proof, it would be feasible for Bob to lie about Alice owning $\address$. In
other words, $\taxAuth$ needs to distinguish between two scenarios:
\begin{inparaenum}[i)]
    \item Alice controls $\address$ and tax evades, or
    \item Bob is lying about Alice owning $\address$.
\end{inparaenum}
Observe that, in the first scenario, Bob \emph{does} know that $\address$ is
controlled by Alice. However, due to the pseudonymity of cryptocurrency
systems, the transaction $\tau$ is not sufficient to prove this fact. Instead,
Bob needs a proof which can only be supplied by Alice, \eg a signature from
Alice which acknowledges $\tau$ or $\address$. However, if Alice tax evades via
$\address$, it is safe to assume that they would not create such proof
and incriminate themselves.

\paragraph{Desiderata.}\label{subsec:desiderata}

The core requirement of a taxation mechanism is to compel taxpayers to declare
their assets, \ie to prohibit tax evasion. However, it is important that we
retain as many good features of existing cryptocurrency systems as possible.
One such core feature is transaction privacy. Bitcoin-like ledgers enhance
privacy via the pseudonymous nature of addresses, while designs like
Zerocash~\cite{SP:BCGGMT14} utilize zero-knowledge schemes to achieve full
anonymity of transactions, \ie hiding the sender, recipient, and amount of each
individual transaction. Our aim is to retain at least Bitcoin levels of privacy
by minimizing the information which is leaked during the taxation process.

Another important aspect is the performance of a taxation mechanism. A
fundamental ingredient of cryptocurrency systems is the seamless transaction
experience. Thus, it is important to allow users to transact at all times,
while also avoiding to introduce significant strain to the system during
taxation. Finally, our mechanisms aim to minimize the amount of (additional)
data which are published to the ledger, since storage in distributed ledgers
costs significantly.

In conclusion, the desiderata of a taxation mechanism are as follows:
\begin{itemize}
    \item \textbf{Tax evasion prevention}: tax evasion, \ie failure of a user
        $\user$ to declare the amount of assets they own, should be either
        detectable by $\taxAuth$ or prohibited by the taxation mechanism.
    \item \textbf{High level of privacy}: the taxation authority $\taxAuth$
        should at most learn the total amount of assets owned by each taxpayer
        at the end of a fiscal year; this information should be leaked only to
        $\taxAuth$ and no additional information should be leaked to any other
        party, apart from the information already leaked by the ledger.
    \item \textbf{Unobstructed operation}: the introduction of a taxation
        mechanism should not result in any period over which the tax compliant
        users are unable to transact.
    \item \textbf{Low performance overhead}: the taxation mechanism should not
        introduce a major performance overhead, in terms of computation and
        storage requirements from the users and the taxation authority.
    \item \textbf{Balanced load}: the computation and storage overhead of
        taxation should be spread over a period of time, rather than
        result in computation spikes.
\end{itemize}
