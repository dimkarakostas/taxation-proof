\section{Properties and Limitations of Cryptocurrency Taxation}\label{sec:taxation}

Our goal is to design a taxation mechanism which retains as much privacy as
possible. A basic requirement is that the user $\user$ declares its assets to
the taxation authority $\taxAuth$.  Specifically, in every cryptocurrency
system, a user $\user$ manages their assets via \emph{addresses}. Each address
$\address$ is associated with a key pair $\keypair$, such that the private key
$\signkey$ is used to claim ownership of the assets, \eg by signing special
messages; in Bitcoin, as well as most cryptocurrency systems, $\address =
\mathsf{H}(\verifykey)$ for some hash function $\mathsf{H}$. Assuming
$[\address_1, \dots, \address_n]$ is the list of all addresses that $\user$
controls, the total amount of assets that it owns is $\assets = \sum_{i=1}^n
\balance(\address_i)$, where $\balance(\address)$ is the (publicly available)
amount of assets controlled by $\address$.  Ideally, this information, \ie
$\assets$, is the only information that is leaked to $\taxAuth$, while ensuring
the user does not ``hide'' any assets.

\paragraph{Limitations.}\label{subsec:limitations}

We showcase the limitations of current systems with a telling example.  Assume
that Alice tax evades, \ie creates a secret, undeclared address $\address$ and
controls some assets $\theta$ in it. As long as $\address$ is kept hidden, \ie
Alice does not use it, it is impossible for any party to identify or prove that
Alice owns $\address$ and, consequently, $\theta$.
Now assume that, in the future, Alice issues a transaction $\tau$ which
sends $\theta$ assets from $\address$ to Bob. If Bob suspects that Alice evaded
taxation for the $\theta$ assets, they might want to report them to the
taxation authority $\taxAuth$ for inspection, \ie they would form a complaint
that Alice owns $\address$ and tax evades via it.
When $\taxAuth$ receives Bob's complaint, it faces a problem. Even if
$\taxAuth$ has proof that Alice did not declare $\address$ for taxation, it
further needs proof that $\address$ is in fact controlled by Alice, \ie a proof
that Alice knows the private key that pertains to $\address$. Without such
proof, it would be feasible for Bob to lie about Alice owning $\address$. In
other words, $\taxAuth$ needs to distinguish between two scenarios:
\begin{inparaenum}[i)]
    \item Alice controls $\address$ and tax evades, or
    \item Bob is lying about Alice owning $\address$.
\end{inparaenum}
Observe that, in the first scenario, Bob \emph{does} know that $\address$ is
controlled by Alice. However, due to the pseudonymity of cryptocurrency
systems, the transaction $\tau$ is not sufficient to prove this fact. Instead,
Bob needs a proof which can only be supplied by Alice, \eg a signature from
Alice which acknowledges $\tau$ or $\address$. However, if Alice tax evades via
$\address$, it is safe to assume that they would not create such proof, which
would effectively incriminate themselves.

\paragraph{Desiderata.}\label{subsec:desiderata}

Although enabling taxation is a step forward, it is crucial that we retain as
many good features of existing cryptocurrency systems as possible. One such
core feature is transaction privacy. Bitcoin-like ledgers enhance privacy via
the pseudonymous nature of addresses, while designs like
Zerocash~\cite{SP:BCGGMT14} utilize zero-knowledge to achieve full anonymity of
transactions, \ie hiding the sender, recipient, and amount of each individual
transaction. Our aim is to retain at least Bitcoin levels of privacy by
minimizing the information which is leaked during the taxation process.

Another important aspect is the performance of a taxation mechanism.  A
fundamental ingredient of cryptocurrency systems is the seamless transaction
experience, thus it is important to allow users to transact at all times, while
also avoiding to introduce significant strain to the system during, \eg, the
taxation period. Finally, any mechanism should aim to minimize the amount of
(additional) data which are published to the ledger, since the storage cost in
distributed ledgers is a core measure of efficiency.

In conclusion, the desiderata of a taxation mechanism are as follows:
\begin{itemize}
    \item \textbf{High level of privacy}: the tax authority $\taxAuth$ should
        at most learn the total amount of assets owned per user at the end of a
        tax year, and no additional information should be leaked to any party
        other than $\taxAuth$, apart from the information already leaked by
        the ledger.
    \item \textbf{Unobstructed operation}: the introduction of a taxation
        mechanism should not introduce any time period over which the (tax
        compliant) users are unable to use their assets.
    \item \textbf{Low performance overhead}: the taxation mechanism should not
        introduce a major performance overhead (\ie computation and storage).
    \item \textbf{Balanced performance load}: the performance overhead should
        be spread over a period of time, rather than result in computation
        spikes.
\end{itemize}
