\subsection{Desiderata}\label{sec:taxation}

In distributed ledger-based currency systems, a user $\user$ manages their
assets via addresses. Each address $\address$ is associated with a key pair
$\keypair$, such that the private key $\signkey$ is used to claim ownership of
the assets, \eg by signing special messages; typically $\address =
\hash(\verifykey)$ for some hash function $\hash$. Each address $\address$ is
associated with a (public) balance $\balance(\address)$ so, given a list
$[\address_1, \dots, \address_n]$ of all addresses that $\user$ controls,
$\user$'s total assets are $\assets = \sum_{i=1}^n \balance(\address_i)$. Our
goal will be to retain as much privacy as possible, so $\assets$ should be the
only information that is leaked to $\taxAuth$, without
de-anonymization of individual transaction data.

To showcase the limitations of current systems, consider the following example.
Assume that Alice tax evades, \ie creates a secret, undeclared address
$\address$ and controls some assets $\asset$ in it. Given the pseudonymous
nature of the ledger, $\address$ cannot be correlated with Alice, until she
uses it. Following, Alice issues a transaction $\tau$ which sends $\theta$
assets from $\address$ to Bob. If Bob suspects that Alice evaded taxation for
these $\theta$ assets, they might want to report her to the authorities for
inspection. However, the complaint should be accompanied by a proof that
$\address$ is controlled by Alice, \ie a proof that Alice knows the private key
associated with $\address$. This is necessary as $\taxAuth$ needs to
distinguish between two scenarios:
\begin{inparaenum}[i)]
    \item Alice controls $\address$ and tax evades;
    \item Bob is lying about Alice owning $\address$.
\end{inparaenum}
In the first scenario, Bob \emph{does} know that $\address$ is controlled by
Alice, but $\tau$ is not sufficient to prove it.
Instead, Bob needs a proof which can only be supplied by Alice, \eg a signature
from Alice which acknowledges $\tau$ or $\address$. However, if Alice tax
evades, naturally she would not create such incriminating proof.

It is important that we retain as many good features of existing ledger systems
as possible. The most notable such feature is transaction privacy, thus our
work considers pseudonymous, Bitcoin-like levels of privacy, and minimizes the
information leaked to the authorities during a tax auditing. Another important
aspect is the mechanism's performance. A fundamental ingredient of payment
systems is the seamless transaction experience, so it is important to allow
users to transact at all times, while also avoiding significant strain during
taxation periods. Finally, our mechanisms aim to minimize the amount of
(additional) published data, since storage in
distributed ledgers is particularly costly.

In summary, the desiderata of our mechanisms are as follows:
\begin{itemize}
    \item \emph{Tax gap identification}: Tax evasion, \ie failure of a user
        $\user$ to declare the amount of assets they own, should be either
        detectable by a tax authority $\taxAuth$, with access to the
        ledger, or render the assets unusable.
    \item \emph{High level of privacy}: $\taxAuth$ should --- at most ---
        learn the total amount of assets owned by each taxpayer at the end of a
        fiscal year; this information should be leaked only to $\taxAuth$ and
        no additional information should be leaked to any other party, apart
        from the information already published on the ledger.
    \item \emph{Unobstructed operation}: The introduction of a taxation
        mechanism should not result in any period during which the --- tax
        compliant --- users are prohibited from transacting.
    \item \emph{Low performance overhead}: The taxation mechanism should not
        introduce a major performance overhead, in terms of computation and
        storage requirements from the users and the taxation authority.
    \item \emph{Balanced load}: The computation and storage overhead of
        taxation should be spread over a period of time, rather than introduce
        performance spikes.
\end{itemize}
