\section{Introduction}\label{sec:introduction}

Decentralized blockchains face a number of pressing challenges.  Scalability,
interoperability, and governance are only a few of the concepts that
researchers have been working on during the past decade, since
Bitcoin~\cite{nakamoto2008bitcoin} first launched. Nevertheless, although
immensely interesting, technical issues are not the only challenges that such
systems face. A common topic of heated discussions around the future of
decentralized blockchains is \emph{adoption}.  Mainstream adoption has been
almost ubiquitously hailed as the way for blockchain systems to mature and,
potentially, appreciate in value. Indeed, as more people build on and use
decentralized ledgers, it is expected that more research opportunities will
arise, helping researchers and entrepreneurs alike evolve the blockchain space.
Eventually, adoption will allow the scientific knowledge and technological
standards of distributed ledgers to mature faster and better.

A core element regarding adoption of every new technology, like blockchains, is
compliance with the existing regulatory frameworks. As early as 2010, questions
on how, or even if, Bitcoin can be a legal investment product have sparked a
heated discussion\footnote{One of the first such investigations is available at
Eric Roberts' webpage:
\url{https://cs.stanford.edu/people/eroberts/courses/cs181/projects/2010-11/Bitcoins/should-bitcoin-be-legal.html}}.
However, due to the fact that the legal framework has been trailing the
technological advancements, established organizations are often hesitant to
adopt or abstain altogether from using the new blockchain technology\footnote{A
primary example is the Electronic Frontier Foundation's step back in accepting
Bitcoin as a donation option in 2011 due to ``complex legal issues'', as
detailed in the following article:
\url{https://www.eff.org/deeplinks/2011/06/eff-and-bitcoin}}. It is thus clear
that, in order for --- decentralized --- blockchains to be widely used, proper
tools should be developed that allow asset owners and custodians to comply with
changes in the legal framework while retaining much needed properties, like
privacy, that the technology offers. Indeed, crypto-assets have drawn a lot of
attention from tax authorities recently, especially after the surge in
Bitcoin's price during 2017.  Exchanges, being the primary platforms for
trading cryptocurrency and shaping the market's prices, have been put on the
spot by authorities across the globe, \eg the U.S. Securities and Exchange
Commission, which as of 2018 requires exchanges to register with
it~\cite{securities2018statement}, and the Australian tax office, which applies
trading stock rules on exchanges~\cite{tax2019statement}.

Our work aims to accelerate the technical discussion needed to incorporate
taxation in decentralized blockchain systems. We explore the properties and
limitations of taxation mechanisms, aiming to implement secure taxation while
maintaining the best level of privacy possible and minimizing the level of
trust put in the taxation authority. First, we describe a ledger which compels
taxpayers to tax their crypto-assets in a privacy-preserving manner, our core
idea being to freeze the assets until they are taxed. Such mechanism can either
be built in new systems or enforced via a hard fork, \ie small but decisive
changes in deployed consensus protocols, and offers privacy which is at least
as good as Bitcoin-like systems. Second, our paper introduces a taxation scheme
directly applicable on existing systems. Our design is based on
Provisions~\cite{CCS:DBBCB15}, a proof of solvency for Bitcoin exchanges, which
we extend in order to allow, on the one hand, users or enterprises to declare
their assets without disclosing their entire transaction history, and, on the
other hand, to enable an authority identify tax evasion.
