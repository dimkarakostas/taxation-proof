\section{Introduction}\label{sec:introduction}

A tax gap~\cite{comission2018taxgaps} is a difference between the reported and
the real tax revenue, for a given jurisdiction and period of time. Research
estimated that the tax gap in the USA was $16.4$\% of revenue
owed~\cite{internal2016federal} between 2008-2010, the total loss throughout
the EU due to the tax gap to €$151.5$ billion in 2015~\cite{murphy2018resources}, while
$\frac{1}{3}$ of taxpayers in the UK under-report their
earnings~\cite{advani2020does} (albeit half of UK's lost taxes are product of a
small, wealthy fraction of misbehaving taxpayers). Therefore, reducing the tax
gaps can significantly enhance the efforts of tax-collecting authorities.

Central bank digital currencies (CBDC) have also come to prominence in recent
years. In the past decade, distributed ledger-based financial systems, which
were kick-started with the creation of Bitcoin~\cite{nakamoto2008bitcoin}, were
accompanied by the increasing digitalization of payments~\cite{bis2011digital}.
CBDCs are the culmination of these trends, enabling fast, cheap, and safe
transactions in fiat assets. Crucially though, although still mostly on a
research stage,\footnote{\url{https://cbdctracker.org} [July 2021]} CBDCs have
caused great concerns on citizens regarding transaction
privacy~\cite{ecb2021cbdcprivacy}.

Our work offers two mechanisms that facilitate tax auditing and the
identification of tax gaps in distributed ledger-based currency systems. The
first is a wrapper around a generic distributed ledger, which enables taxpayers
to declare their assets directly to the authorities, while undeclared assets
are frozen. The second is a proof mechanism that enables the sender of some
assets to prove, in a privacy-preserving manner, whether the transferred assets
have been taxed. Both mechanisms are examples of programmable money (also referred to as smart money~\cite{AHA}),
where currency is programmed to be transferable under a suitable set of  circumstances or its transfer has specific implications.
