\section{Introduction}\label{sec:introduction}

Decentralized blockchains face a number of pressing challenges.
Scalability, interoperability, and governance are only a few of the concepts
that researchers have been working on during the decade since 2009, when
Bitcoin~\cite{nakamoto2008bitcoin} first launched. Nevertheless, technical
issues, although immensely interesting, are not the only challenges that such
systems face. A common theme that arises in discussions and articles regarding
the future of decentralized blockchains is \emph{adoption}.
Mainstream adoption has been almost ubiquitously hailed as the way for
blockchain systems to mature and, potentially, appreciate in value. Indeed, as
more people build on and use decentralized ledgers, it is expected that more
research opportunities will arise, while helping researchers and entrepreneurs
alike. Eventually, adoption should allow the scientific knowledge and
technological standards of distributed ledgers to mature faster and better.

A core element regarding adoption of every new technology, like blockchains, is
compliance with the existing regulatory frameworks. As early as 2010, questions
on how, or even if, Bitcoin can be a legal assets have sparked a heated
discussion\footnote{One of the first such discussions is available at Eric
Roberts' webpage:
\url{https://cs.stanford.edu/people/eroberts/courses/cs181/projects/2010-11/Bitcoins/should-bitcoin-be-legal.html}}.
However, due to the fact that the legal framework has been trailing the
technological advancements, in more than one occasions established
organizations have been hesitant or abstained from using the new blockchain
technology\footnote{A primary example is the Electronic Frontier Foundation's
step back in accepting Bitcoin as a donation option in 2011 due to ``complex
legal issues'', as detailed in the following article:
\url{https://www.eff.org/deeplinks/2011/06/eff-and-bitcoin}}. It is thus clear
that, in order for --- decentralized --- blockchains to be widely used, proper
tools should be developed that allow asset owners and custodians to comply with
the changes in the legal framework, while retaining the much needed properties,
like privacy, that the technology offers.

Crypto-assets have drawn a lot of attention from the tax authorities recently,
especially after the surge in Bitcoin's price during 2017. Additionally,
exchanges, being the primary platforms for trading cryptocurrency and shaping
the market's prices, have been put on the spot by authorities across the
globe. Primary examples are the U.S. Securities and Exchange Commission, which
as of 2018 require exchanges to register with
it~\cite{securities2018statement}, as well as the Australian tax office, which
applies the trading stock rules on exchanges~\cite{tax2019statement}.

Our work aims to accelerate the technical discussion on cryptocurrency
taxation.  We explore the needed properties and limitations of such mechanisms,
aiming to maintain the best level of privacy possible, both for the exchanges
and their users, as well as minimize the level of trust that needs to be
assumed of the taxation authority. Next, we describe a ledger which allows
the taxpayers to declare their crypto-assets to in a
privacy-preserving manner. Our core idea is to freeze the assets until they are
taxed, a radical change which can either be integrated in new systems or
enforced on existing systems via a hard fork. Notably, the privacy of our
mechanism is at least as good as Bitcoin-like systems. Finally, our paper
introduces a taxation scheme for enterprises, which is directly applicable on
systems like Bitcoin. Our design is based on Provisions~\cite{CCS:DBBCB15}, a
proof of solvency mechanism for Bitcoin exchanges, which we extend in order to
allow enterprises to declare their assets without necessarily disclosing their
entire transaction history. Importantly, we describe how a tax authority can
use our scheme to identify whether an enterprise fails to declare some of its
assets.
