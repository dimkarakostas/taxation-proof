\paragraph{Related work.}\label{sec:related}

The main area of research which touches our paper is the proof of solvency line
of work. The first proof of solvency scheme for Bitcoin exchanges was proposed
by Maxwell~\cite{wilcox2014proving}. However, this scheme leaks the total
amount of both assets and liabilities of the exchange, while, more importantly,
enabling an attack that allows the exchange to hide assets, as detailed by
Doerner \etal in Zeroledge~\cite{doernerzeroledge}. Zeroledge also proposed a
proof system with which exchanges can prove properties about their holdings
without revealing sensitive information. Our taxation mechanism for enterprises
is based on Provisions by Dagher \etal~\cite{CCS:DBBCB15}, a
privacy-preserving proof of solvency mechanism for Bitcoin exchanges. Based on
a number of Sigma protocols, Provisions allows exchanges to prove solvency in
zero-knowledge, \ie without the need to reveal the addresses or the amount of
assets that it controls. Similarly, Agrawal \etal~\cite{C:AgrGanMoh18} describe
a proof of solvency using SNARKS, thus achieving better performance compared to
Provisions, although assuming a trusted setup. Importantly, this work enables
proof of solvency for not only pay-to-public-key but also
pay-to-public-key-hash addresses, which is the most common type of addresses.
Our work describes an extension to Provisions, although it can also be applied
to Agrawal \etal

A holistic approach regarding auditable ledgers was taken in the design of
zkLedger by Narula \etal~\cite{EPRINT:NarVasVir18}. The taxable ledger of
Section~\ref{sec:taxchain} also defines a ledger which by design enables audits
and taxation, as opposed to the approach of
Section~\ref{sec:provisions-extension}, which attempts to apply a mechanism on
top of existing systems. However, our approach is much simpler than zkLedger
regarding the changes needed in existing designs in order to accommodate
taxation.
