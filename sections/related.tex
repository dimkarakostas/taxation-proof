\paragraph{Related work.}\label{sec:related}

Our work is closer to the proof of solvency line of work. The first proof of
solvency scheme for Bitcoin exchanges, proposed by
Maxwell~\cite{wilcox2014proving}, leaks the total amount of both assets and
liabilities of the exchange. More importantly, it enables an attack that allows
the exchange to hide assets, as detailed by Doerner \etal in
Zeroledge~\cite{doernerzeroledge}, which also proposed a privacy-preserving
system that allows exchanges to prove properties about their holdings.
Provisions, by Dagher \etal~\cite{CCS:DBBCB15}, is also a proof of solvency
mechanism for Bitcoin exchanges based on Sigma protocols, which allows
exchanges to prove solvency in zero-knowledge, \ie without the need to reveal
the addresses or the amount of assets that it controls.  Similarly, Agrawal
\etal~\cite{C:AgrGanMoh18} describe a proof of solvency using SNARKS, achieving
better performance compared to Provisions although assuming a trusted setup.
This work enables proof of solvency for not only pay-to-public-key but also
pay-to-public-key-hash addresses, which is the most common type of addresses.
Our work describes an extension to Provisions, although it also applies to
Agrawal \etal Finally, a holistic approach regarding auditable ledgers was
taken in the design of zkLedger by Narula \etal~\cite{EPRINT:NarVasVir18}. The
taxable ledger of Section~\ref{sec:taxchain} defines a ledger which by design
enables audits and taxation, although our approach is much simpler than
zkLedger regarding the changes needed in existing designs in order to
accommodate taxation.
