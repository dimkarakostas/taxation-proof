\subsection{Related work}\label{sec:related}

Literature offers various works on auditing of distributed ledger-based assets.
A holistic approach is taken in zkLedger~\cite{narula2018zkledger}, which
combines a permissioned ledger with zero-knowledge proofs to create a
tamper-resistant, verifiable ledger of transactions.
PRCash~\cite{EPRINT:WKCC18} also employs a permissioned ledger and offers a
regulation mechanism that restricts the total amount of assets a user can
receive anonymously for a period of time. Also Garman
\etal~\cite{FC:GarGreMie16} propose an anonymous ledger, which can enforce
specific transaction policies. In our paper, Section~\ref{sec:taxchain} aims at
offering a simpler design, which can be more easily integrated in existing
pseudonymous distributed ledgers, compared to the aforementioned works. Another
interesting research thread considers proofs of solvency. The first such scheme
for Bitcoin exchanges, proposed by Maxwell~\cite{wilcox2014proving}, leaks the
total amount of both assets and liabilities of the exchange; more importantly,
it enables an attack that allows the exchange to hide assets, as detailed by in
Zeroledge~\cite{doernerzeroledge}, which also proposed a privacy-preserving
system that allows exchanges to prove properties about their holdings.
Provisions~\cite{CCS:DBBCB15} is a zero-knowledge proof of solvency mechanism
for Bitcoin exchanges, based on Sigma protocols \ie without the need to reveal
the addresses or the amount of assets that an exchange controls. Similarly,
Agrawal \etal~\cite{C:AgrGanMoh18} describe a proof of solvency which achieves
better performance compared to Provisions, although assuming a trusted setup.
The mechanism of Section~\ref{sec:provisions-extension} extends
Provisions and is also applicable to~\cite{C:AgrGanMoh18}.
