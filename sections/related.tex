\subsubsection{Related work.}\label{sec:related}

To the best of our knowledge, our work is the first to discuss taxation for
decentralized blockchain systems, as well as the freezing of assets as a
mechanism to enforce users to declare their assets. A holistic approach
regarding auditable ledgers is zkLedger by Narula
\etal~\cite{EPRINT:NarVasVir18}. The taxable ledger of
Section~\ref{sec:taxchain} defines a ledger which by design enables audits and
taxation, although our approach is much simpler than zkLedger regarding the
changes needed in existing designs in order to accommodate taxation.  Our
second solution builds upon and extends existing research on proofs of
solvency. The first such scheme for Bitcoin exchanges, proposed by
Maxwell~\cite{wilcox2014proving}, leaks the total amount of both assets and
liabilities of the exchange. More importantly, it enables an attack that allows
the exchange to hide assets, as detailed by Doerner \etal in
Zeroledge~\cite{doernerzeroledge}, which also proposed a privacy-preserving
system that allows exchanges to prove properties about their holdings.
Provisions, by Dagher \etal~\cite{CCS:DBBCB15}, is also a proof of solvency
mechanism for Bitcoin exchanges based on Sigma protocols, which allows
exchanges to prove solvency in zero-knowledge, \ie without the need to reveal
the addresses or the amount of assets that it controls. Similarly, Agrawal
\etal~\cite{C:AgrGanMoh18} describe a proof of solvency using SNARKS, achieving
better performance compared to Provisions although assuming a trusted setup;
this work enables proof of solvency for not only pay-to-public-key but also
pay-to-public-key-hash addresses, which is the most common type of addresses.
The mechanism of Section~\ref{sec:provisions-extension} is an extension to
Provisions which also applies to Agrawal \etal
