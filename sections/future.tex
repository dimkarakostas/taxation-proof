\section{Conclusion}\label{sec:conclusion}

Our work offers a programmable money
approach for authorities to audit the citizens' tax returns and create
a tax-gap counter-incentive: undeclared fund transfers are programmed to
be frozen in the ledger. We identify a number of limitations and
desiderata and present two basic designs, which can act as a stepping stone for
more concrete solutions. Our mechanisms can be employed by different tax
authorities and be applied on different ledger designs. Naturally, to
efficiently utilize it on a global scale for decentralized systems, like
Bitcoin, tax authorities of all countries would need to collaborate, an
assumption which seems infeasible in our current fragmented landscape.
Nevertheless, a single country's sovereign could deploy it as a feature of, for
example, a central bank digital currency.  Particular points of interest for
future work are the effect of freezing on user experience, as well as the
storage overhead. Additionally, our scheme considers pseudonymous systems;
future work could explore fully anonymous applications, which utilize
zero-knowledge schemes to achieve cryptographic-grade
transaction anonymity. Finally, an
interesting direction is the design of incentive schemes that motivate the
system's adoption and reduce the dependence on enforcement by the authorities.
