\section{Conclusion}\label{sec:conclusion}

Our work offers a mechanism for authorities to audit the citizens' tax returns
for distributed ledger-based assets. We identify a number of limitations and
desiderata and present two basic designs, which can act as a stepping stone for
more concrete solutions. Our mechanisms can be employed by different tax
authorities and be applied on different ledger designs. Naturally, to
efficiently utilize it on a global scale for decentralized systems, like
Bitcoin, tax authorities of all countries would need to collaborate, an
assumption which seems infeasible in our current fragmented landscape.
Nevertheless, a single country's sovereign could deploy it as a feature of, for
example, a central bank digital currency.  Particular points of interest for
future work are the effect of freezing on user experience, as well as the
storage overhead. Additionally, our scheme aims at pseudonymous levels of
privacy; however, future work could explore applications for designs like
Zerocash~\cite{SP:BCGGMT14}, which utilize zero-knowledge schemes to achieve
perfect transaction anonymity.  An additional interesting research direction is
the design of incentive schemes that motivate the system's adoption and reduce
the dependence on enforcement by the authorities.
