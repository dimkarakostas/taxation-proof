\section{Conclusion}\label{sec:conclusion}

The current paper focuses on the issue of mainstream adoption of decentralized
blockchain assets, specifically tackling the problem of taxation of
crypto-assets. After identifying the requirements, as well as the inherent
limitations, of such mechanisms, we construct a scheme building on the
well-known Provisions mechanism for the auditing of the solvency of
crypto-exchanges.

Our mechanism, although enabling a user to identify whether the assets they
receive from an exchange have been properly taxed, is constrained in a number
of ways. First, as described in Section~\ref{subsec:tax-design}, our mechanism
is more effective as more users utilize it. However, our design lacks
incentives which would motivate a user to use it. Second, although proving the
tax evasion to a third party in the standard setting is impossible (cf.
Section~\ref{subsec:tax-limitations}), perhaps stronger assumptions could lead
to more optimistic results. For instance, if every exchange was require to use
specially-crafted addresses, then it could be possible that a user who
identifies a tax evasion incident be able to report it (including a proof) to
the authority. Future work will explore such designs, as well as all privacy
implications.
