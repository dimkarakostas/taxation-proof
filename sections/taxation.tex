\section{The Taxation Mechanism}\label{sec:taxation}

Conjectures:
\begin{itemize}
    \item It is impossible to achieve taxation properties without proof of solvency
\end{itemize}

\subsection{The Properties of a Taxation mechanism}\label{subsec:taxation-properties}

\begin{proposition}\label{prop:tax-proof-impossibility}
    It is impossible to prove tax evasion to a third party without cooperation
    from the (malicious) exchange.
\end{proposition}
\begin{proof}
    We will describe two scenarios, where
    \begin{inparaenum}
        \item the exchange tax evades and
        \item the user lies
    \end{inparaenum}
    and show that they are indistinguishable.

    Scenario $1$ (malicious exchange $\exchange$, honest user $\user$):
    \begin{enumerate}
        \item $\user$ gives to $\exchange$ an address $d$ and requests a
            withdrawal of $\theta$ assets
        \item $\exchange$ publishes the transaction $tx = \langle s, d, \theta
            \rangle$, where $s$ is an address which is not taxed, \ie is a tax
            evasion address.
    \end{enumerate}

    Scenario $2$ (honest exchange $\exchange$, malicious user $\user$):
    \begin{enumerate}
        \item $\user$ controls some address $s$, which holds $\theta$ assets
        \item $\user$ creates a new address $d$ and publishes a transaction
            $tx = \langle s, d, \theta \rangle$
    \end{enumerate}

    Due to Bitcoin's pseudonymity, there is no connection between an address
    and the physical entity which owns it, \ie $\exchange$ and $\user$ in our
    case. Therefore, given only the transaction $tx$, scenarios $1$ and $2$ are
    indistinguishable.
\end{proof}

Proposition~\ref{prop:tax-proof-impossibility} shows that, without cooperation
from the exchange, it is impossible to prove that a tax evasion has occurred.
However, it is safe to assume that a tax-evading exchange would not cooperate
anyway in proving its tax evasion, in other words self-incrimination.

Still, a user can identify the tax evasion, although not being able to prove it
to a third party. For instance, in the Scenario $1$ above, a user can request a
proof that the assets which originate from address $s$ are properly taxed. In
case of tax evasion, then it should be impossible to produce such proof,
therefore the exchange would simply refuse to respond. Thus, the exchange's
refusal to reply is an admission of guilt.

We can now use this observation to describe a setting where the exchange admits
tax evasion to the tax authority itself. First, the user is allowed to ask for
taxation proof \emph{after} the withdrawal has occurred, \ie after the exchange
has committed to owning the address $s$. Second, the exchange does not know
whether the user will request a taxation proof, \ie cannot a priori choose
whether to use a tax evading address or not. Therefore, the tax authority can
impersonate a user of the exchange and, after properly completing a withdrawal,
request a proof. Now, if the exchange used a tax evading address it will have
admitted tax evasion to the authority itself.
