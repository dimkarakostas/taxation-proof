\section{The Taxation Mechanism}\label{sec:taxation}

In this section we explore the necessary properties and core limitations of a
taxation mechanism.

\paragraph{Desiderata.}\label{subsec:taxation-desiderata}

\begin{itemize}
    \item \textbf{Privacy of Assets Under Management}: the total amount of assets
        that the exchange manages should not be leaked to any party, except the
        taxation authority $\taxAuth$.
    \item \textbf{Address Privacy}: the ownership of an address $\address$ that
        the exchange controls should not be leaked to any party except a
        user who receives assets from $\address$.
    \item \textbf{Tax Evasion Identification}: $\taxAuth$ should be able to
        acquire proof of tax evasion, \ie whether the exchange declares to
        $\taxAuth$ less assets than it controls.
    \item \textbf{Provable Address Taxation}\comment{is this necessary?}: the
        exchange should be able to prove the taxation of one of its addresses
        to any party without leaking any additional information (except the
        address itself).
\end{itemize}

\paragraph{Impossibilities of a Taxation Mechanism.}\label{subsec:taxation-properties}

Having established the minimal requirements that we need from a taxation
mechanism, we now explore what is actually achievable. Therefore, we describe
the following impossibility results, which will help establish a basis of
discussion.

\begin{proposition}\label{prop:tax-identification-impossibility}
    It is impossible to identify tax evasion, unless the exchange reveals or
    uses an address which controls some undeclared assets.
\end{proposition}

It is easy to show that Proposition~\ref{prop:tax-identification-impossibility}
holds. Specifically, consider an exchange $\exchange$ which creates a secret
address $\address$ and sends some assets $\theta$ to it. Next, as long as the
exchange does not reveal or use $\address$ in any way, it is impossible for
another party to identify or prove that $\exchange$ owns $\address$ and,
consequently, $\theta$.

\begin{proposition}\label{prop:tax-proof-impossibility}
    It is impossible to prove tax evasion to a third party using only a
    transaction.
\end{proposition}
We will describe two scenarios, where
\begin{inparaenum}[i)]
    \item the exchange tax evades and
    \item the user lies,
\end{inparaenum}
and will show that they are indistinguishable.

Scenario $1$ (malicious exchange $\exchange$, honest user $\user$):
\begin{enumerate}
    \item $\user$ gives to $\exchange$ an address $d$ and requests a
        withdrawal of $\theta$ assets
    \item $\exchange$ publishes the transaction $tx = \langle s, d, \theta
        \rangle$, where $s$ is a private address of the exchange, \ie is
        not taxed.
\end{enumerate}

Scenario $2$ (honest exchange $\exchange$, malicious user $\user$):
\begin{enumerate}
    \item $\user$ controls some address $s$, which holds $\theta$ assets
    \item $\user$ creates a new address $d$ and publishes a transaction
        $tx = \langle s, d, \theta \rangle$
\end{enumerate}

Due to Bitcoin's pseudonymity, there is no connection between an address
and the physical entity which owns it, \ie $\exchange$ or $\user$ in our
case. Therefore, given only transaction $tx$, the scenarios $1$ and $2$ are
indistinguishable.

Proposition~\ref{prop:tax-proof-impossibility} thus shows that, without
cooperation from the exchange, it is impossible to prove that a tax evasion has
occurred. However, assuming help from the exchange is equivalent to asking
from a malicious exchange to incriminate itself; it is safe to assume that it
would not participate in such actions.

Still, the tax evasion is identifiable, due to the exchange's refusal for
cooperation. For instance, in the Scenario $1$ above, a user can request a
proof that assets originating from address $s$ are properly taxed. If the
exchange refuses to reply, it is safe to assume that this address is used for
tax evasion, since the exchange has no incentive for not providing such proof.

\paragraph{Design.}

In order to design a taxation mechanism we keep in mind that, in case an
exchange $\exchange$ evades taxation for some address, the taxation authority
identifies the evasion. In order to achieve this, we utilize the fact that tax
evasion is identifiable, as described above.

First, we require that a user is allowed to request taxation proof \emph{after}
the withdrawal has occurred, \ie after $\exchange$ has committed to owning the
address $s$. Following Proposition~\ref{prop:tax-identification-impossibility},
the identification occurs when $\exchange$ uses a tax evading address to send
some funds to a user, \ie upon a withdrawal by the user. Given
Proposition~\ref{prop:tax-proof-impossibility}, the user who performs the
withdrawal should be the tax authority $\taxAuth$ itself.

Second, $\exchange$ should not know whether the user will request a taxation
proof. Naturally this is necessary since, if $\exchange$ knows a priori that a
user would request a taxation proof, it would use properly taxed addresses.

Given the above requirements, $\taxAuth$ can covertly become a user of the
exchange, perform a withdrawal, and then request a taxation proof. If the
exchange uses a tax evading address for this withdrawal then it will not be
able to provide a valid taxation proof, hence it will have admitted tax evasion
to the authority itself.
