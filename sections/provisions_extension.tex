\section{A Tax-Auditing Extension for Provisions}\label{sec:provisions-extension}

We now build a tax auditing mechanism for existing ledgers based on
Provisions~\cite{CCS:DBBCB15}. The goal of this mechanism is to enable a user
to identify whether the assets they receive from a sender $\exchange$ have been
properly declared. This is achieved via a taxation proof, which created
\emph{after} the transaction is issued, \ie after $\exchange$ commits to owning
the assets. If $\exchange$ fails to provide such proof, then it can safely
assumed that it evaded taxation. Therefore, to audit $\exchange$, $\taxAuth$
poses as a counter-party in a transaction.  It is important that $\exchange$
does not know a priori whether a taxation proof will be requested as, in that
case, it could decide beforehand to transfer ``clean'' assets.
To build this proof mechanism we rely on Provisions~\cite{CCS:DBBCB15},
particularly its \emph{proof of assets}. Our scheme comprises of two simple
protocols, which $\exchange$ runs with the taxation authority and their
counter-party respectively. As we show, our protocols retain the privacy
guarantees of Provisions.

Provisions is a privacy-preserving auditing mechanism for Bitcoin exchanges.
Using Provisions a party can verify that a (cooperating) Bitcoin exchange is
solvent, \ie possesses enough assets to cover the liabilities towards its
users. In order to achieve this, Provisions defines three protocols:
\begin{inparaenum}[i)]
    \item proof of assets,
    \item proof of liabilities, and
    \item proof of solvency.
\end{inparaenum}
% The first protocol commits the exchange --- in a zero-knowledge fashion ---
% to the total amount of assets it possesses. Similarly, the second protocol
% commits it to the liabilities towards its clients, such that each client can
% verify that the exchange has included his/her deposits in the collective
% proof.  Finally, the proof of solvency proves that the exchange's assets are
% equal or surpass its liabilities.
Our work is only interested in the assets owned by the exchange, thus we focus
on the proof of assets. All proofs are considered under a group $G$ of prime
order $q$ with fixed public generators $g, h$. The proof of assets considers
the following:
\begin{itemize}
    \item $\text{\textbf{PK}} = \{y_1, \dots, y_n \}$: the (anonymity) set of public keys;
    \item $s_i$: a bit such that, if the exchange controls $y_i$, \ie if it possesses the private key of $y_i$, then $s_i = 1$, otherwise $s_i = 0$;
    \item $\balance(y_i)$: the amount of assets that the address corresponding to $y_i$ controls;
    \item $\assets = \sum_{i = 1}^n s_i \cdot \balance(y_i)$: the amount of assets that the exchange controls;
    \item $b_i = g^{\balance(y_i)}$: a commitment to the balance of $y_i$.
\end{itemize}
The exchange publishes the Pedersen commitments~\cite{C:Pedersen91} for each $s_i \cdot
\balance(y_i), s_i$:
\begin{align}
    p_i = b_i^{s_i} \cdot h^{v_i} = g^{\balance(y_i) \cdot s_i} \cdot h^{v_i} \label{eq:balance-commit} \\
    l_i = y_i^{s_i}h^{t_i} \label{eq:ownership-commit} \\
    l_i = g^{\hat{x}_i}h^{t_i} \label{eq:privkey-commit}
\end{align}
where $v_i, t_i \in \mathbb{Z}_q$ are chosen at random,
$x_i$ is the private key for $y_i$, and $\hat{x}_i = x_i \cdot s_i$.
The exchange proves knowledge of values $s_i, v_i, t_i, \hat{x}_i$ for every $i
\in [1, n]$ via a $\Sigma$-protocol, such that conditions
(\ref{eq:balance-commit}), (\ref{eq:ownership-commit}), and
(\ref{eq:privkey-commit}) are satisfied.
% for completeness, the Provisions' proof of assets protocol is included in
% Appendix~\ref{subsec:provisions-assets-proof}.

\paragraph{The Tax-Auditing Extension}\label{subsec:tax-authority-proto}
First, $\taxAuth$ requests the total amount of assets that $\exchange$
controls, \ie the value $\assets$, to verify that $\exchange$'s commitments
correspond to $\assets$. We obtain the condition
$Z_\assets = \prod_{i = 1}^n p_i = g^{\assets} \cdot h^v$,
where $v = {\sum_{i = 1}^n v_i}$, is a (publicly
computable) Pedersen commitment to $\exchange$'s assets. Given that $\taxAuth$
knows $\assets$, $\exchange$ needs only to prove knowledge of a value $v$, such
that this condition is satisfied. This is done via the Schnorr
protocol~\cite{C:Schnorr89} of Figure~\ref{fig:taxation_auth_proto}, which
guarantees privacy as described in Lemma~\ref{thm:tax-auth-proto}.

\begin{figure}[h]
\begin{mdframed}

\begin{center}
    \textbf{Asset Taxation Protocol} $\taxationProto$
\end{center}

    \begin{itemize}
        \item Public data: $g, h, Z_\assets = \prod_{i = 1}^n p_i$
        \item Verifier's input from prover: $\assets$
        \item Prover's input: $v = \sum_{i = 1}^n v_i$
    \end{itemize}

    \begin{enumerate}
        \item The prover ($\exchange$) chooses $r \xleftarrow{\$} \mathbb{Z}_q$
            and sends $\lambda = h^r$ to the verifier ($\taxAuth$).
        \item The verifier replies with a challenge $c \xleftarrow{\$} \mathbb{Z}_q$.
        \item The prover responds with $\theta = r + c \cdot v$.
        \item The verifier accepts if $h^\theta \stackrel{?}{=} \lambda \cdot (Z_\assets \cdot g^{-\assets})^c$.
    \end{enumerate}

\end{mdframed}
\caption{
    Tax-auditing between $\exchange$ (prover) and $\taxAuth$ (verifier).
}
\label{fig:taxation_auth_proto}
\end{figure}

\begin{lemma}\label{thm:tax-auth-proto}
    For public values $g, h, Z_\assets$, the protocol $\taxationProto$ is an
    honest-verifier zero-knowledge argument of knowledge of quantity $v$
    satisfying
    $Z_\assets = \prod_{i = 1}^n p_i = g^{\assets} \cdot h^v$ for $i \in [1, n]$.
\end{lemma}

\paragraph{Address Verification}\label{subsec:user-verification-proto}
The second part of our taxation proof enables the tax auditing of a specific
address by every user $\user$. $\exchange$ now needs to prove two conditions to
$\user$:
\begin{inparaenum}[i)]
    \item for some $i \in [1, n]$, the public key $y_i$ (which is published as
        part of the Provisions scheme) corresponds to the address from which
        $\user$ receives their assets;
    \item the corresponding bit $s_i$ for $y_i$ in the commitment condition
        (\ref{eq:ownership-commit}) is $s_i = 1$.
\end{inparaenum}
The first condition can be easily proven by providing $\user$ with an index
$i$, such that $\user$ confirms that the address in question is equal to the
hash of $y_i$. To prove the second condition, we observe that, for $s_i = 1$,
condition (\ref{eq:ownership-commit}) becomes
$l_i = y_ih^{t_i}$.
Therefore, $\exchange$ needs only to prove knowledge of $t_i$, such that this
statement is satisfied, which can be achieved via the Schnorr protocol
of Figure~\ref{fig:taxation_verification_proto}; as before,
Lemma~\ref{thm:user-proto} formalizes the necessary level of privacy based on
the security properties of the Schnorr protocol.

\begin{figure}[h]
\begin{mdframed}

\begin{center}
    \textbf{Address Verification Protocol} $\taxationAddressProto$
\end{center}

    \begin{itemize}
        \item Public data: $h$, $(y_i, l_i)$ for $i \in [1, n]$
        \item Verifier's input from prover: $i$
        \item Prover's input: $t_i$
    \end{itemize}

    \begin{enumerate}
        \item The prover ($\exchange$) chooses $r \xleftarrow{\$} \mathbb{Z}_q$
            and sends $\lambda = h^r$ to the verifier.
        \item The verifier replies with a challenge $c \xleftarrow{\$} \mathbb{Z}_q$.
        \item The prover responds with $\theta = r + c \cdot t_i$.
        \item The verifier accepts if $h^\theta \stackrel{?}{=} \lambda \cdot (l_i \cdot y_i^{-1})^c$.
    \end{enumerate}

\end{mdframed}
\caption{
    Address verification between $\exchange$ (prover) and a user $\user$ (verifier).
}
\label{fig:taxation_verification_proto}
\end{figure}

\begin{lemma}\label{thm:user-proto}
    For public values $h$ and $(y_i, l_i)$, the protocol
    $\taxationAddressProto$ is an honest-verifier zero-knowledge argument of
    knowledge of quantity $t_i$ satisfying $l_i = y_ih^{t_i}$.
\end{lemma}

% Finally, both protocols can be turned into non-interactive zero-knowledge
% (NIZK) proofs of knowledge in the random oracle model by using the
% Fiat-Shamir transformation~\cite{C:FiaSha86}.
