\section{A Tax-Auditing Extension for Provisions}\label{sec:provisions-extension}

We now build a tax auditing mechanism for existing ledgers based on
Provisions~\cite{CCS:DBBCB15}. The goal of this mechanism is to enable all
payment recipients to verify whether the assets used by a sender $\exchange$ in
a transaction have been properly declared to the authority $\taxAuth$. This is
achieved in two stages, first with an asset declaration stage that involves
$\taxAuth$ and second with a payer address auditing protocol, which is created
in tandem with the transaction that pays a recipient, and after $\exchange$
commits to owning the assets. If $\exchange$ fails to provide such proof, the
implication is that $\exchange$ performs tax evasion.  To build this
mechanism we rely on Provisions~\cite{CCS:DBBCB15}, particularly its
\emph{proof of assets}. Our scheme comprises of two simple protocols, which
$\exchange$ runs with the taxation authority and their counter-party
respectively. As we show, our protocols retain Provisions' privacy guarantees.

Provisions is a privacy-preserving auditing mechanism for Bitcoin exchanges.
Using Provisions a party can verify that a (cooperating) Bitcoin exchange is
solvent, \ie possesses enough assets to cover the liabilities towards its
users. In order to achieve this, Provisions defines three protocols:
\begin{inparaenum}[i)]
    \item proof of assets,
    \item proof of liabilities, and
    \item proof of solvency.
\end{inparaenum}
Our work is only concerned in the assets owned by the exchange, thus we focus
on the proof of assets. All proofs are considered under a group $G$ of prime
order $q$ with fixed public generators $g, h$. The proof of assets considers
the following:
\begin{itemize}
    \item $\text{PK} = \{y_1, \dots, y_n \}$: the total (anonymity) set of public keys;
    \item $s_i$: a bit such that, if the exchange controls $y_i$, \ie if it possesses the private key of $y_i$, then $s_i = 1$, otherwise $s_i = 0$;
    \item $\balance(y_i)$: the amount of assets that the address corresponding to $y_i$ controls;
    \item $\assets = \sum_{i = 1}^n s_i \cdot \balance(y_i)$: the amount of assets that the exchange controls;
    \item $b_i = g^{\balance(y_i)}$: a binding (but not hiding) commitment  to the balance of $y_i$.
\end{itemize}
The exchange publishes the Pedersen commitments~\cite{C:Pedersen91} for each $s_i \cdot
\balance(y_i), s_i$:
\begin{align}
    p_i = b_i^{s_i} \cdot h^{v_i} = g^{\balance(y_i) \cdot s_i} \cdot h^{v_i} \label{eq:balance-commit} \\
    l_i = y_i^{s_i}h^{t_i} =  g^{\hat{x}_i}h^{t_i} \label{eq:ownership-commit}
\end{align}
where $v_i, t_i \in \mathbb{Z}_q$ are chosen at random,
$x_i$ is the private key for $y_i$, and $\hat{x}_i = x_i \cdot s_i$.

\subsubsection*{Asset Declaration.}\label{subsec:tax-authority-proto}
 $\exchange$ declares the total amount of assets
it controls, \ie the value $\assets$,
to  $\taxAuth$ who verifies  that $\exchange$'s commitments
correspond to $\assets$. We obtain the condition
$Z_\assets = \prod_{i = 1}^n p_i = g^{\assets} \cdot h^v$,
where $v = {\sum_{i = 1}^n v_i}$, is a (publicly
computable) Pedersen commitment to $\exchange$'s assets. Given that $\taxAuth$
knows $\assets$, $\exchange$ needs only to prove knowledge of a value $v$, such
that this condition is satisfied. This is done via the Schnorr
protocol~\cite{C:Schnorr89} of Figure~\ref{fig:taxation_auth_proto}, which
guarantees privacy (cf. Lemma~\ref{thm:tax-auth-proto}).

\myhalfbox{Asset Declaration Protocol $\taxationProto$}{white!40}{white!10}{
    Public data: $g, h, Z_\assets = \prod_{i = 1}^n p_i$

    Verifier's input from prover: $\assets$

    Prover's input: $v = \sum_{i = 1}^n v_i$
    \begin{enumerate}
        \item The prover ($\exchange$) chooses $r \xleftarrow{\$} \mathbb{Z}_q$
            and sends $\lambda = h^r$ to the verifier ($\taxAuth$).
        \item The verifier replies with a challenge $c \xleftarrow{\$} \mathbb{Z}_q$.
        \item The prover responds with $\theta = r + c \cdot v$.
        \item The verifier accepts if $h^\theta \stackrel{?}{=} \lambda \cdot (Z_\assets \cdot g^{-\assets})^c$.
    \end{enumerate}
}{\label{fig:taxation_auth_proto} Tax-auditing between $\exchange$ (prover) and $\taxAuth$ (verifier).}


\begin{lemma}\label{thm:tax-auth-proto}
    For public values $g, h$ and $Z_\assets$, the protocol $\taxationProto$ is an
    honest-verifier zero-knowledge argument of knowledge of quantity $v$
    satisfying
    $Z_\assets = \prod_{i = 1}^n p_i = g^{\assets} \cdot h^v$ for $i \in [1, n]$.
\end{lemma}

\subsubsection*{Payer Address Auditing.}\label{subsec:user-verification-proto}
The second part of our taxation proof enables the auditing of a specific
address of a payer $\exchange$, when a payment is made to another
user $\user$. $\exchange$ will prove two conditions to $\user$:
\begin{inparaenum}[i)]
    \item for some $i \in [1, n]$, the public key $y_i$ (which is published as
        part of the Provisions scheme) corresponds to the address from which
        $\user$ receives their assets;
    \item the corresponding bit $s_i$ for $y_i$ in the commitment condition
        (\ref{eq:ownership-commit}) is $s_i = 1$.
\end{inparaenum}
The first condition can be easily proven by providing $\user$ with an index
$i$, such that $\user$ confirms that the address in question is equal to the
hash of $y_i$. To prove the second condition, we observe that, for $s_i = 1$,
$p_i = g^{\balance(y_i)}h^{v_i}$ and
$l_i = y_ih^{t_i}$.
Therefore, $\exchange$ needs only to prove knowledge of $t_i$ and $v_i$, such that this
statement is satisfied, which can be achieved via the Schnorr protocol
of Figure~\ref{fig:taxation_verification_proto}, its privacy properties formalized in
Lemma~\ref{thm:user-proto}.

\myhalfbox{Address Verification Protocol $\taxationAddressProto$}{white!40}{white!10}{
    Public data: $h$, $(y_i, l_i), \balance(y_i)$ for $i \in [1, n]$

    Verifier's input from prover: $i$

    Prover's input: $t_i$
    \begin{enumerate}
        \item The prover ($\exchange$) chooses $r_1, r_2 \xleftarrow{\$} \mathbb{Z}_q$
            and sends $\lambda_1 = h^{r_1}, \lambda_2 = h^{r_2}$ to the verifier.
        \item The verifier replies with a challenge $c \xleftarrow{\$} \mathbb{Z}_q$.
        \item The prover responds with $\theta_1 = r_1 + c \cdot t_i$,
        $\theta_2 = r_2 + c \cdot v_i$.
        \item The verifier accepts if $h^{\theta_1} \stackrel{?}{=} \lambda_1 \cdot (l_i \cdot y_i^{-1})^c$
        and $h^{\theta_2} \stackrel{?}{=} \lambda_2 \cdot (p_i \cdot g^{-\balance(y_i)})^c$.
    \end{enumerate}
}{\label{fig:taxation_verification_proto} Address verification between $\exchange$ (prover) and a user $\user$ (verifier).}

\begin{lemma}\label{thm:user-proto}
    For public values $g, h$ and $y_i, l_i, p_i, \balance(y_i)$, the protocol
    $\taxationAddressProto$ is an honest-verifier zero-knowledge argument of
    knowledge of quantities $t_i, v_i$ satisfying $l_i = y_ih^{t_i}$ and $p_i =
    g^{\balance(y_i)}h^{v_i}$ respectively.
\end{lemma}

Finally, both protocols can be turned into non-interactive zero-knowledge
(NIZK) proofs of knowledge in the random oracle model by using the
Fiat-Shamir transformation~\cite{C:FiaSha86}.
