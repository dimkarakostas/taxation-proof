\section{A Taxation Extension for Provisions}\label{sec:provisions-extension}

In this section we extend Provisions~\cite{CCS:DBBCB15} with a taxation proof.
Specifically, we first give a brief overview of the Provisions scheme,
importantly the \emph{proof of assets} which is of interest in our line of
work, and then define two simple protocols for the taxation authority and the
exchange's client respectively. We show that our protocols retain the privacy
guarantees of Provisions, as well as prevent the exchange from evading taxation
for some of its assets.

\subsection{Overview of Provisions}\label{subsec:provisions-overview}

The core motivation behind Provisions was to provide a
\emph{privacy-preserving} auditing mechanism for Bitcoin exchanges. Using
Provisions a user can verify that a (cooperating) Bitcoin exchange is solvent,
\ie possesses enough assets to cover the liabilities towards its client-users.
In order to achieve this, the Provisions mechanism defines three protocols:
\begin{inparaenum}[i)]
    \item proof of assets,
    \item proof of liabilities, and
    \item proof of solvency.
\end{inparaenum}
The basic idea is that the proof of assets commits the exchange --- in a
zero-knowledge fashion --- to the total amount of assets it possesses, the
proof of liabilities commits it to the liabilities towards its clients, such
that each client can verify that the exchange has included his/her deposits in
the proof, and finally the proof of solvency proves that the assets are enough
to cover the liabilities.

Our work is only interested in the assets owned by the exchange, thus the
proofs of liabilities and solvency are not of interest, rather we focus on the
proof of assets. Provisions' proofs are all considered under a group $G$ of
prime order $q$ with fixed public generators $g, h$. For the proof of assets,
we have the following parameters:
\begin{itemize}
    \item $\text{\textbf{PK}} = \{y_1, \dots, y_n \}$: the (anonymity) set of public keys;
    \item $s_i$: a bit such that, if the exchange controls $y_i$, \ie if it possesses the private key of $y_i$, then $s_i = 1$, otherwise $s_i = 0$;
    \item $\text{bal}(y_i)$: the amount of assets that the address corresponding to $y_i$ controls;
    \item $\assets = \sum_{i = 1}^n s_i \cdot \text{bal}(y_i)$: the total amount of assets that the exchange controls;
    \item $b_i = g^{\text{bal}(y_i)}$: a commitment to the balance of $y_i$.
\end{itemize}

The exchange publishes two Pedersen commitments. First, it commits to each $s_i
\cdot \text{bal}(y_i)$:
\begin{align}
    p_i = b_i^{s_i} \cdot h^{v_i} = g^{\text{bal}(y_i) \cdot s_i} \cdot h^{v_i} \label{eq:balance-commit}
\end{align}
where $v_i \in \mathbb{Z}_q$ is chosen at random.
Second, it commits to each $s_i$:
\begin{align}
    l_i = y_i^{s_i}h^{t_i} \label{eq:ownership-commit} \\
    l_i = g^{\hat{x}_i}h^{t_i} \label{eq:privkey-commit}
\end{align}
where $x_i$ is the private key for $y_i$, $\hat{x}_i = x_i \cdot s_i$, and $t_i
\in \mathbb{Z}_q$ is chosen at random.

In order to complete the proof, Provisions defines a $\Sigma$-protocol in order
to prove knowledge of values $s_i, v_i, t_i, \hat{x}_i$ for every $i \in [1,
n]$ such that conditions (\ref{eq:balance-commit}),
(\ref{eq:ownership-commit}), and (\ref{eq:privkey-commit}) are satisfied; the
proof of assets protocol is included in
Appendix~\ref{subsec:provisions-assets-proof} for completeness.

\subsection{The Asset Taxation Protocol}\label{subsec:tax-authority-proto}

The first part of our mechanism is the taxation authority's responsibility.
Specifically, the authority requests from the exchange the value of the total
number of assets that it controls and needs to verify that the provided amount
is correct.

This verification can be easily done by utilizing Provisions' proof of
assets mechanism. First, observe that:
\begin{align}
    Z_\assets = \prod_{i = 1}^n p_i = g^{\assets} \cdot h^v \label{eq:assets}
\end{align}
where $v = {\sum_{i = 1}^n v_i}$, is a --- publicly computable --- Pedersen
commitment to the total amount of assets that the exchange controls. Second,
the value $\assets$ is available to the taxation authority, since it is
submitted by the exchange in order to be taxed.  Therefore, the exchange needs
only to prove knowledge of a value $v$, such that condition (\ref{eq:assets}) is
satisfied. To achieve this, the exchange and the authority run the Schnorr
protocol~\cite{C:Schnorr89} defined in Figure~\ref{fig:taxation_auth_proto}.

\begin{figure}[h]
\begin{mdframed}

\begin{center}
    \textbf{Asset Taxation Protocol} $\taxationProto$
\end{center}

    \begin{itemize}
        \item Public data: $g, h, Z_\assets = \prod_{i = 1}^n p_i$
        \item Verifier's input from prover: $\assets$
        \item Prover's input: $v = \sum_{i = 1}^n v_i$
    \end{itemize}

    \begin{enumerate}
        \item The prover ($\exchange$) chooses $r \xleftarrow{\$} \mathbb{Z}_q$
            and sends $\lambda = h^r$ to the verifier ($\taxAuth$).
        \item The verifier replies with a challenge $c \xleftarrow{\$} \mathbb{Z}_q$.
        \item The prover responds with $\theta = r + c \cdot v$.
        \item The verifier accepts if $h^\theta \stackrel{?}{=} \lambda \cdot (Z_\assets \cdot g^{-\assets})^c$.
    \end{enumerate}

\end{mdframed}
\caption{
    The Asset Taxation Protocol which is run between the exchange $\exchange$
    (prover) and the taxation authority $\taxAuth$ (verifier).
}
\label{fig:taxation_auth_proto}
\end{figure}

\begin{theorem}
    For public values $g, h, Z_\assets = \prod_{i = 1}^n p_i$ for $i \in [1,
    n]$, the protocol $\taxationProto$ is an honest-verifier zero-knowledge
    argument of knowledge of quantity $v$ satisfying condition (\ref{eq:assets}).
\end{theorem}
\begin{proof}
    The proof follows directly from the properties of Schnorr's protocol. \qed
\end{proof}

\subsection{The User Verification Protocol}\label{subsec:user-verification-proto}

The second part of our taxation proof is the user verification. Whenever a user
receives a transfer from the exchange, \eg during a withdrawal of some assets,
the user needs to verify that the assets they receive have been properly taxed.

The exchange now needs to prove two conditions:
\begin{inparaenum}[i)]
    \item for some $i \in [1, n]$, the public key $y_i$ of the proof of assets
        corresponds to the address from which the assets are sent to the user,
    \item the corresponding bit for $y_i$ is $s_i = 1$.
\end{inparaenum}
The first condition can be easily proven by giving to the user an index $i$,
such that the user confirms that the address from which they receive the assets
is equal to the hash of $y_i$. To prove the second condition, observe that, for
$s_i = 1$, statement (\ref{eq:ownership-commit}) now becomes:
\begin{align}
    l_i = y_ih^{t_i} \label{eq:owned-address-commit}
\end{align}
Therefore, the exchange only needs to prove knowledge of a $t_i$, such that
statement (\ref{eq:owned-address-commit}) is satisfied; this is achieved via
the Schnorr protocol defined in Figure~\ref{fig:taxation_verification_proto}.

\begin{figure}[h]
\begin{mdframed}

\begin{center}
    \textbf{Address Verification Protocol} $\taxationAddressProto$
\end{center}

    \begin{itemize}
        \item Public data: $h$, $(y_i, l_i)$ for $i \in [1, n]$
        \item Verifier's input from prover: $i$
        \item Prover's input: $t_i$
    \end{itemize}

    \begin{enumerate}
        \item The prover ($\exchange$) chooses $r \xleftarrow{\$} \mathbb{Z}_q$
            and sends $\lambda = h^r$ to the verifier.
        \item The verifier replies with a challenge $c \xleftarrow{\$} \mathbb{Z}_q$.
        \item The prover responds with $\theta = r + c \cdot t_i$.
        \item The verifier accepts if $h^\theta \stackrel{?}{=} \lambda \cdot (l_i \cdot y_i^{-1})^c$.
    \end{enumerate}

\end{mdframed}
\caption{
    The Address Verification Protocol which is run between the exchange $\exchange$
    (prover) and the user (verifier).
}
\label{fig:taxation_verification_proto}
\end{figure}

\begin{theorem}
    For public values $h$ and $(y_i, l_i)$, the protocol
    $\taxationAddressProto$ is an honest-verifier zero-knowledge argument of
    knowledge of quantity $t_i$ satisfying condition (\ref{eq:owned-address-commit}).
\end{theorem}
\begin{proof}
    Again the proof follows from the Schnorr protocol's properties. \qed
\end{proof}
