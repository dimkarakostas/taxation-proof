\begin{abstract}
    A major bottleneck against the widespread adoption and usage of
    decentralized blockchain systems is the distance between the needs of the
    enterprises and the regulatory bodies. A primary example of this tension is
    the trade-off between user privacy and the taxation of cryptocurrency
    assets. In this work we discuss the problem of enabling taxation mechanisms
    in cryptocurrency systems. We identify the required properties and
    limitations of such mechanisms and propose two designs.  First, we describe
    a ledger which enables taxation by default, which is achieved via only
    small changes in existing designs.  Second, we provide an extension which
    enables enterprises to declare their assets and can be built on top of most
    existing cryptocurrencies. Both solutions retain a high level of privacy,
    by ensuring that the only information which is leaked to the taxation
    authority, in addition to information already leaked by the underlying
    ledger, is the total amount of a user's assets.
\end{abstract}
