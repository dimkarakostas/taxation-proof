\begin{abstract}
%    A major bottleneck against the widespread adoption and usage of
%    decentralized blockchain systems is the distance between the needs of the
%    enterprises and the regulatory bodies. A primary example of this tension is
%    the trade-off between user privacy and the taxation of cryptocurrency
%    assets.
In this work we discuss the problem of facilitating taxation mechanisms
    in cryptocurrency systems. We identify the required properties of such mechanisms
    including preserving privacy and tax evasion protection and outline the
    challenges of implementing mechanisms in existing cryptocurrencies.  
    We  describe two techniques that such mechanisms can be realised. First,  we describe
    a ledger which enables taxation as a built-in feature. This is achieved via only
    small changes in existing ledger designs.  Second, we describe an extension 
    mechanism which  can be built on top of most
    existing cryptocurrencies without any modification in the underlying ledger.  
    Both solutions provide a high level of privacy
     ensuring that no other information 
        (beyond the information which is already leaked 
    by the underlying
    ledger)  
    is leaked to any party except 
    for a
    taxation authority, 
     which learns the total amount of a user's assets that are to be taxed. 
\end{abstract}
