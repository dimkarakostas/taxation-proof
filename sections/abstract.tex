\begin{abstract}
    A major bottleneck against the widespread adoption and usage of
    decentralized blockchain systems is the distance between the needs of the
    enterprises and the regulatory bodies. A primary example of this tension is
    the trade-off between user privacy and the taxation of cryptocurrency
    assets. In this work we explore the problem of enabling taxation mechanisms
    in cryptocurrency systems. We identify the required properties and
    limitations of such mechanisms and propose two designs. First, we provide
    an extension, which can be built on top of most existing cryptocurrencies,
    which enables enterprises to declare their assets. Second, we describe a
    ledger which enables taxation by default via a small number of changes in
    existing designs. Both solutions retain a high level of privacy by ensuring
    the only information which leaks to the taxation authority (and no other
    third party) is the total amount of assets which a user owns.
\end{abstract}
