\section{A Taxable Ledger}\label{sec:taxchain}

In this section we describe a ledger which enables taxation by design. Our
mechanism is generic, such that existing distributed ledgers can incorporate it
via simple changes in the consensus protocol. A taxable ledger enables a user
$\user$ to declare the amount of crypto-assets they own to a taxation authority
$\taxAuth$. In achieving this, we aim at a high level of privacy. Ideally, we
want only the total amount of assets that $\user$ owns to be leaked to
$\taxAuth$. However, given the underlying structure of the ledger, it is
possible that additional information is leaked; \eg in a pseudonymous setting
like Bitcoin, one or more of the addresses might be linked to the user who owns
them.

\paragraph{Assumptions.}

We assume that $\taxAuth$ has a list $P$ of all taxpayers.
Additionally, we assume the existence of taxation periods which last $d$ time;
\eg a taxation period lasts $1$ year, at the end of which the taxpayers are
required to declare their assets. Naturally, both assumptions are in line with
how real-world tax systems operate. Finally, we assume a key
$\keypair_\taxAuth$ which is controlled by and identifies $\taxAuth$ and is
published on the ledger.
The core idea behind our scheme is to freeze assets which are not accounted
for. In particular, assume a taxation period $d_i$. At the end of the taxation
period, all users are required to declare their assets with $\taxAuth$.
However, as we saw in Section~\ref{sec:taxation}, unless we assume a complete
lack of privacy, it is impossible to account assets that a user hides in a
newly-created address. However, what we can do is freeze assets until they have
been accounted for.

\paragraph{Design.}

After the taxation period ends and all assets are frozen, $\taxAuth$ enables
users to declare their assets, such that when a user declares their assets
they are unfrozen. When a taxpayer $\user$ wishes to move frozen assets, \ie
either to unfreeze their own or receive frozen assets from another party, they
create a new key pair $\keypair_\user$ and corresponding address
$\address_\user$. Then they send $\address_\user$ to $\taxAuth$. Next,
$\taxAuth$ certifies $\address_\user$, by issuing the signature $\sig =
\algosign(\address_\user, \signkey_\taxAuth)$, and returns $\sig$ to $\user$.
The tuple $\address_\user^{t} = \langle \address_\user, \sig \rangle$ is the
``taxation'' address which is used by the user to receive frozen assets.
Additionally, $\taxAuth$ maintains a mapping of taxpayers to taxation
addresses, \ie for every taxpayer $\user$ it holds a list $A_\user$ of all
taxation addresses certified for $\user$ such as $\address_\user$.
A transaction $\tau = \langle \address_{source}, \address_{dest}, x \rangle$
which moves $x$ frozen assets from the address $\address_{source}$ is valid
only if $\address_{dest} = \langle \address, \sig \rangle:
\algoverify(\address, \sig, \verifykey_\taxAuth) = 1$. Therefore, the miners
will accept transactions that unfreeze assets only as long as said assets are
transferred to a taxation address.
Finally, $\taxAuth$ can compute the amount of assets for which $\user$ should
be taxed as $\assets_\user := \sum_{i=1}^{n} \balance(A_{\user}[i])$,
$n$ being the total number of $\user$'s taxation addresses. We note that the
system can be extended for multiple taxation authorities, \eg to accommodate a
list of countries. If $\taxAuth$ is a federation of authorities, each
identified by a single key, the list of keys for all taxation authorities is
published instead of $\keypair_\taxAuth$. A taxpayer can then certify its
address with the authority of their country of residence and thus declare their
assets only to this authority.

\paragraph{Challenges.}

The first immediate effect of the taxation mechanism is the taxation addresses
themselves. Standard pay-to-public-key-hash
addresses are $25$ bytes.
However, taxation addresses are significantly larger, due to the certification
signature of $\taxAuth$; for instance, using ECDSA signatures in the DER format
results in additional $72$ bytes, thus making taxation addresses $99$ bytes
long. Fortunately, taxation addresses are expected to be used, in order to
declare the crypto-assets, only once, thus the overall storage cost should not
be significant.
Another important consideration regards the treatment of the frozen assets.
Given the statute of limitations, it might be required that
$\taxAuth$ maintains its taxation private key, as well as the mapping for
certified addresses per users, for a significant period\footnote{For
instance, in the USA the IRS would need to keep this data forever, since there
is no statute of limitations in cases of false, fraudulent, or missing reports:
\url{https://tax.findlaw.com/tax-problems-audits/what-is-the-irs-statute-of-limitations-or-deadline-for-action-on-.html}}.
% This might introduce additional costs in the log keeping that $\taxAuth$ needs
% to do.

\paragraph{A taxable Bitcoin.}

We showcase our design via a taxable variation of Bitcoin, $\taxBtc$. $\taxBtc$
is initially parameterized by the public key of the taxation authority
$\keypair_\taxAuth$, which is part of the ledger's genesis block. This key is
used to both certify taxation addresses; in order to update it and use a new
key $\keypair_\taxAuth'$, $\taxAuth$ simply signs $\verifykey_\taxAuth'$ with
$\signkey_\taxAuth$ and publishes it on the ledger.
We assume that a taxation period lasts $52560$ blocks, \ie roughly $1$ calendar
year, so block $52560$ and its multiples are ``taxation'' blocks.  When a
taxation block is issued, all \emph{non-taxation} addresses which control
$\taxBtc$ are frozen.

The freezing introduces a number of complications which need to be addressed.
First, the liveness property (cf.~\cite{EC:GarKiaLeo15}) of a transaction may
be affected. For instance, a transaction which spends from a non-taxation
address will be rejected if it is not included in a block before the freezing
takes place. We sidestep this issue by enabling users to use taxation addresses
before the freezing period; thus the liveness guarantees of the blockchain
apply unconditionally for taxation addresses, whereas close to the freezing
period would be affected. Second, it is possible that multiple competing
taxation blocks are created, \eg multiple blocks which extend the $52559$th
block, $\taxAuth$ needs to pick only one and certify it; afterwards, the
certified taxation block cannot be reverted, thus acting as a ``checkpoint'' in
the chain.

In order to spend $\taxBtc$ from a frozen address $\address \in A$, the user
issues a transaction $\transaction = \langle \address, \address', \assets
\rangle$. $\transaction$ is valid if, assuming $\address' = \langle
\hat{\address}, \sig \rangle$, it holds that $\algoverify(\hat{\address}, \sig,
\verifykey_\taxAuth) = 1$; any transaction which does not conform to this rule,
\ie which attempts to send funds to a non-taxation address, is invalid and
rejected by the miners. The address $\hat{\address}$ can be any standard
Bitcoin address, therefore in order to spend from it the user can issue a
regular Bitcoin transaction.

We observe that $\taxBtc$ covers the desiderata proposed in
Section~\ref{subsec:desiderata}. Regarding privacy, although $\taxAuth$ can
de-anonymize the set of $\taxBtc$ users at a specific point in time, \ie when the
system freezes, it does not have any advantage during all other periods, when
the users employ standard Bitcoin addresses and transactions. Additionally,
similarly to standard Bitcoin addresses, third parties cannot obtain
information regarding the identity of a taxation address's owner, as long as
the signature employed by $\taxAuth$ to certify the taxation addresses does not
leak such information (which, for typical signature schemes like ECDSA, it does
not).

In terms of performance, a user can transact with their assets always
effortlessly, as long as they use taxation addresses in order to receive or
unfreeze assets during the taxation period. Importantly, the users can certify
and use their addresses ahead of the freezing, thus the certification load is
spread over a longer period of time, \eg a few days or weeks.
