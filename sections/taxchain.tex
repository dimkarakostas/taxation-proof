\section{A Taxable Ledger}\label{sec:taxchain}

In this section we describe a ledger which enables taxation. Our mechanism is
generic enough, such that existing distributed ledgers can incorporate it via a
change in the consensus protocol. In Section~\ref{subsec:tax-bitcoin} we
demonstrate the flexibility of our mechanism by describing a taxation-friendly
version of Bitcoin.

\subsection{The Taxation Consensus Extension}\label{subsec:tax-ledger-extension}

Our mechanism aims to enable a user $\user$ of a distributed ledger to declare
the amount of crypto-assets they own to a taxation authority $\taxAuth$.

In achieving this, we target a high level of privacy. Ideally, we want only the
total amount of assets that $\user$ owns to be leaked to $\taxAuth$. However,
given the underlying structure of the ledger, it is possible that additional
information is leaked; \eg in a pseudonymous setting like Bitcoin, one or more
of the addresses might be linked to the user who owns them.

\paragraph{Assumptions.}

We assume that the tax authority $\taxAuth$ has a list $P$ of all taxpayers.
Additionally, we assume the existence of taxation periods which last $d$ time;
\eg a taxation period lasts $1$ year, at the end of which the taxpayers are
required to declare their assets. Naturally, both assumptions are in line with
how real-world tax systems operate.

The core idea behind our scheme is to freeze assets which are not accounted
for. In particular, assume a taxation period $d_i$. At the end of the taxation
period, all users are required to declare their assets with $\taxAuth$.
However, as we saw in Section~\ref{sec:taxation}, unless we assume a complete
lack of privacy, it is impossible to account assets that a user hides in a
newly-created address. However, what we can do is freeze assets until they have
been accounted for.

\paragraph{Design.}

After the taxation period ends and all assets are frozen, $\taxAuth$ needs to
provide a method for users to declare their assets. Additionally, there should
be established a mechanism such that, when a user declares their assets, then
the assets are unfrozen.

Thus, every taxpayer $\user$ creates a new key pair $\keypair_\user$ and sends
the public key $\verifykey_\user$ to $\taxAuth$. Next, $\taxAuth$ generates the
address $\address_\user$ which corresponds $\verifykey_\user$. Finally, it
collects all these special, ``taxation'' addresses in a list $I$ and publishes
it.  When $\user$ wants to declare their assets, they generate their taxation
address $\address_\user$ and send their assets to it. The protocol ensures that
the miners accept only transactions which move frozen assets to an address in
$I$, thus assets are unfrozen only when they are taxed.

In order to find the amount of assets for which $\user$ should be taxed,
$\taxAuth$ checks the amount of assets that have been sent to $\address_\user$.

We note that a user $A$ could send some assets to the taxation address of a
different user $B$, thus forcing $B$ to be taxed for them; however, in doing
so, $A$ would also actually transfer ownership of the assets to $B$.

\paragraph{Challenges.}

\begin{itemize}
    \item The list $I$ is very expensive in terms of storage. Specifically, it
        contains information for every taxpayer, who are typically in the range
        of millions for most countries.
    \item The taxation declaration is done at the same time for all users.
        Therefore, the end of every taxation period will bring a very big load
        of transactions to the system, which might result in significant delays
        until they are all processed by the miners.
    \item The amount of assets that every taxpayer controls is leaked to
        everybody. Although only $\taxAuth$ knows the correlation between
        taxation addresses and users, the information that a single user
        controls \eg $x$ amount of assets is still significant. A possible
        solution to this is to allow each user to send a list of keys to
        $\taxAuth$, enabling them to split their assets to a number of taxation
        addresses.
\end{itemize}

\subsection{Enabling Taxation on Bitcoin}\label{subsec:tax-bitcoin}
