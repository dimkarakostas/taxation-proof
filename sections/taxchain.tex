\section{A Taxable Ledger}\label{sec:taxchain}

In this section we describe a ledger with a built-in taxation mechanism. Our
design is generic enough for existing ledgers to incorporate it via minimal
changes in consensus. A taxable ledger enforces a user $\user$ to declare the
amount of crypto-assets they own to a taxation authority $\taxAuth$, elseways
the assets are unusable. In achieving this, we aim to leak to $\taxAuth$ only
the total amount of assets that $\user$ owns at a specific point. However,
additional information may be leaked by the ledger itself, \eg in a
pseudonymous setting like Bitcoin addresses might be linked to the user who
controls them.

\paragraph{Assumptions.}

We assume that $\taxAuth$ has a list of all taxpayers and
the existence of taxation periods which last a pre-specified amount of time $d$;
\eg a taxation period may last $1$ calendar year, at the end of which taxpayers need
to declare their assets. We note that both assumptions are in line with
real-world tax systems. Also we assume a key $\keypair[\taxAuth]$, controlled by
and identifying $\taxAuth$, which is published on the ledger.  As we saw in
Section~\ref{sec:taxation}, unless we assume a complete lack of privacy, it is
impossible to account assets that a user hides in a newly-created address.
However, what we can do is freeze assets until they are accounted for, \ie a
taxpayer proves their ownership.

\paragraph{Design.}

After the taxation period ends and all assets are frozen, $\taxAuth$ enables
users to declare their assets, such that each asset which is correctly declared
is unfrozen. When a taxpayer $\user$ wishes to move frozen assets, \ie either
to unfreeze them or receive frozen assets from another party, they create a new
key pair $\keypair[\user]$ and the corresponding address $\address_\user$ and
send $\address_\user$ to $\taxAuth$. Next, $\taxAuth$ certifies
$\address_\user$ by issuing the signature $\sig = \algosign(\address_\user,
\signkey_\taxAuth)$, which it returns to $\user$.  The tuple
$\address_\user^{t} = \langle \address_\user, \sig \rangle$ is the certified
taxation address which is used by the user to receive frozen assets.
Additionally, $\taxAuth$ maintains a mapping of taxpayers to taxation
addresses, \ie for every taxpayer $\user$ it holds a list $A_\user$ of all
certified taxation addresses that $\user$ requested.  A transaction $\tau =
\langle \address_{source}, \address_{dest}, x \rangle$, which moves $x$ frozen
assets from the address $\address_{source}$, is valid only if $\address_{dest}
= \langle \address, \sig \rangle: \algoverify(\address, \sig,
\verifykey_\taxAuth) = 1$. Therefore, the miners accept transactions that
unfreeze assets only as long as said assets are transferred to a certified
taxation address.  Finally, $\taxAuth$ can compute the amount of assets for
which $\user$ should be taxed as $\assets_\user := \sum_{i=1}^{n}
\balance(A_{\user}[i])$, $n$ being the total number of $\user$'s taxation
addresses. We note that the system can be extended for multiple taxation
authorities, \eg to accommodate a list of countries. In that case $\taxAuth$ is
a federation of authorities, each identified by a single key, and the list of
taxation authority keys is published to the ledger. A taxpayer can then certify
its address with and declare its assets to the authority of their country of
residence only.

\paragraph{Challenges.}

The first effect of the taxation mechanism stems from the certification of
taxation addresses.  Although standard pay-to-public-key-hash addresses are
$25$ bytes, taxation addresses may be significantly larger, due to the
certification signature of $\taxAuth$. For instance, ECDSA signatures in the
DER format result in $72$ additional bytes, thus making taxation addresses $99$
bytes long. Fortunately, taxation addresses are expected to be used only once,
in order to declare the crypto-assets, thus the overall storage cost should not
be significant.  Another important consideration regards the private state of
the taxation authority; given the statute of limitations, $\taxAuth$ might need
to maintain its taxation private key and the mapping of certified addresses for
a significant period\footnote{For instance, in the USA the IRS would need to
keep this data forever, since there is no statute of limitations in cases of
false, fraudulent, or missing reports:
\url{https://tax.findlaw.com/tax-problems-audits/what-is-the-irs-statute-of-limitations-or-deadline-for-action-on-.html}}.
% This might introduce additional costs in the log keeping that $\taxAuth$
% needs to do.

\subsubsection{A taxable Bitcoin.}

We showcase our design as a taxable variation of Bitcoin which we denote
$\taxBtc$. $\taxBtc$ is initially parameterized by the public key of the
taxation authority $\keypair[\taxAuth]$, which is part of the ledger's genesis
block.  $\taxAuth$ can update its public key by simply signing a new key
$\verifykey_\taxAuth'$ with $\signkey_\taxAuth$ and publishing it on the
ledger. A taxation period lasts $52560$ blocks, \ie roughly $1$
calendar year, so block $52560$ and its multiples are ``taxation'' blocks.
When a taxation block is issued, all $\taxBtc$ controlled by \emph{non-certified
taxation} addresses are frozen.

Freezing complicates the system in a number of ways. First, the liveness
property of a transaction (cf.~\cite{EC:GarKiaLeo15}) may be affected. For
instance, a transaction which spends from a non-taxation address will be
rejected, if it is created before but published after a taxation block. We
sidestep this issue by enabling users to use taxation addresses before the
freezing period, hence the liveness guarantees of the ledger apply
unconditionally on taxation addresses. Second, it is possible that multiple
competing taxation blocks are created, \eg multiple blocks which extend the
$52559$th block. Therefore, $\taxAuth$ needs to pick only one and certify it.
Afterwards, the certified taxation block cannot be reverted, thus acting as a
``checkpoint''.

In order to spend $\taxBtc$ from a frozen address $\address$, the user issues a
transaction $\transaction = \langle \address, \address', \assets \rangle$.
$\transaction$ is valid if, assuming $\address' = \langle \hat{\address}, \sig
\rangle$, it holds that $\algoverify(\hat{\address}, \sig, \verifykey_\taxAuth)
= 1$; any transaction which does not conform to this rule, \ie which attempts
to send funds to a non-certified taxation address, is invalid and rejected by
the miners. The object $\hat{\address}$ can be a standard Bitcoin address,
therefore in order to spend from it the user can issue a regular Bitcoin
transaction.

We observe that $\taxBtc$ covers the desiderata proposed in
Section~\ref{subsec:desiderata}. Regarding privacy, although $\taxAuth$ can
de-anonymize the set of $\taxBtc$ users at a specific point in time, \ie when
the system freezes, the users can employ standard Bitcoin addresses and
transactions during all other periods. Additionally, similarly to standard
Bitcoin addresses, third parties cannot obtain information regarding the
identity of a taxation address's owner, as long as the certification signature
does not leak it - a standard feature of signature schemes like ECDSA.

In terms of performance, a user can transact with their assets
effortlessly, as long as they use taxation addresses in order to receive or
unfreeze assets during the taxation period. Importantly, the users can certify
and use their addresses ahead of the freezing, thus the certification load is
spread over a longer period of time, \eg a few days or weeks.
