\section{A Taxable Ledger}\label{sec:taxchain}

In this section we describe a ledger which enables taxation. Our mechanism is
generic enough, such that existing distributed ledgers can incorporate it via a
change in the consensus protocol.

A taxable ledger aims to enable a user $\user$ to declare the amount of
crypto-assets they own to a taxation authority $\taxAuth$. In achieving this,
we aim at a high level of privacy. Ideally, we want only the total amount of
assets that $\user$ owns to be leaked to $\taxAuth$. However, given the
underlying structure of the ledger, it is possible that additional information
is leaked; \eg in a pseudonymous setting like Bitcoin, one or more of the
addresses might be linked to the user who owns them.

\paragraph{Assumptions.}

We assume that the tax authority $\taxAuth$ has a list $P$ of all taxpayers.
Additionally, we assume the existence of taxation periods which last $d$ time;
\eg a taxation period lasts $1$ year, at the end of which the taxpayers are
required to declare their assets. Naturally, both assumptions are in line with
how real-world tax systems operate. Finally, we assume a key
$\keypair_\taxAuth$ which is controlled by and identifies $\taxAuth$ and is
published on the ledger.

The core idea behind our scheme is to freeze assets which are not accounted
for. In particular, assume a taxation period $d_i$. At the end of the taxation
period, all users are required to declare their assets with $\taxAuth$.
However, as we saw in Section~\ref{sec:taxation}, unless we assume a complete
lack of privacy, it is impossible to account assets that a user hides in a
newly-created address. However, what we can do is freeze assets until they have
been accounted for.

\paragraph{Design.}

After the taxation period ends and all assets are frozen, $\taxAuth$ needs to
provide a method for users to declare their assets. Additionally, there should
be established a mechanism such that, when a user declares their assets, then
the assets are unfrozen.

When a taxpayer $\user$ wishes to move frozen assets (either unfreeze their own
or receive frozen assets from a third party), they create a new key pair
$\keypair_\user$ and the corresponding address $\address_\user$. Then they send
$\address_\user$ to $\taxAuth$. Next, $\taxAuth$ certifies $\address_\user$, by
issuing the signature $\sig = \algosign(\address_\user, \signkey_\taxAuth)$,
and returns $\sig$ to $\user$. The tuple $\address_\user^{t} = \langle
\address_\user, \sig \rangle$ is the ``taxation'' address which is used by the
user to receive frozen assets. Additionally, $\taxAuth$ maintains a mapping of
taxpayers to taxation addresses, \ie for every taxpayer $\user$ it holds a list
$A_\user$ of all taxation addresses certified for $\user$ such as
$\address_\user$.

A transaction $\tau = \langle \address_{source}, \address_{dest}, x \rangle$
which moves $x$ frozen assets from the address $\address_{source}$ is valid
only if $\address_{dest} = \langle \address, \sig \rangle:
\algoverify(\address, \sig, \verifykey_\taxAuth) = 1$. Therefore, the miners
will accept transactions that unfreeze assets only as long as said assets are
transferred to a taxation address.

Afterwards, $\taxAuth$ can find the amount of assets for which $\user$ should
be taxed by computing $\assets_\user := \sum_{i=1}^{n} \balance(A_{\user}[i])$,
where $n$ is the total number of $\user$'s taxation addresses.

\emph{Note:} The above system can be extended for multiple taxation
authorities, \eg to accommodate a list of countries. In that case, $\taxAuth$
is a federation of authorities, each identified by a single key, and
the list of keys for all taxation authorities is published instead of
$\keypair_\taxAuth$. A taxpayer can then certify its address with the authority
of their country of residence and thus declare their assets only to this
authority.

\paragraph{Challenges.}

The first immediate effect of the taxation mechanism is the taxation addresses
themselves. Standard pay-to-public-key-hash
addresses\footnote{\url{https://en.bitcoin.it/wiki/Address}} are $25$ bytes.
However, taxation addresses are significantly larger, due to the certification
signature of $\taxAuth$; for instance, using ECDSA signatures in the DER format
results in additional $72$ bytes, thus making taxation addresses $99$ bytes
long. Fortunately, taxation addresses are expected to be used, in order to
declare the crypto-assets, only once, thus the overall storage cost should not
be significant.

Another important consideration regards the treatment of the frozen assets.
Given the statute of limitations, it might be required that the tax authority
$\taxAuth$ maintains its taxation private key, as well as the mapping for
certified addresses per users, for a significant amount of time\footnote{For
instance, in the US, the IRS would need to keep this data forever, since there
is no statute of limitations in cases of false, fraudulent, or missing reports
(cf.
\url{https://tax.findlaw.com/tax-problems-audits/what-is-the-irs-statute-of-limitations-or-deadline-for-action-on-.html})}.
This might introduce additional costs in the log keeping that $\taxAuth$ needs
to do.
