\section{Tax Auditable Distributed Ledger}\label{sec:taxchain}

In this section we describe a ledger with a built-in tax auditing mechanism.
Our design is generic, such that existing ledgers can incorporate it with
minimal changes in the underlying consensus protocol. An \emph{auditable
ledger} enforces a user $\user$ to declare the amount of assets they own to a
taxation authority $\taxAuth$, with failure to do so rendering the assets
unusable. We achieve this while leaking to $\taxAuth$ only the total amount of
assets that $\user$ owns at a specific point in time, \eg the end of a fiscal
year. We note that we consider only pseudonymous ledgers, so potentially
de-anonymizable data may be published on the ledger, \eg addresses which may be
linked to the user who controls them.

We assume that $\taxAuth$ holds a list of all taxpayers and is identified by a
key $(\signkey_{\taxAuth}, \verifykey_{\taxAuth})$. Also there exist taxation
periods, which last for a pre-specified amount of time $d$. For example, a
taxation period may last $1$ calendar year, at the end of which taxpayers need
to declare their assets to the authorities.

The core idea is that assets unaccounted for, at the end of the taxation
period, are frozen, until their owners declare them to the authority.
Specifically, at the end of a taxation period, all assets are frozen. To
unfreeze an asset, a taxpayer $\user$ declares it to $\taxAuth$ as follows.

First, $\user$ creates a new key pair $(\signkey_{\user}, \verifykey_{\user})$
and the corresponding address $\address_{\user}$ and sends $\address_{\user}$
to $\taxAuth$ as part of a KYC process.  Next, $\taxAuth$ certifies $\address_{\user}$ by issuing the
signature $\sig = \algosign(\address_{\user}, \signkey_{\taxAuth})$, which it
gives to $\user$. The tuple $\address_{\user}^{t} = \langle \address_{\user},
\sig \rangle$ is the \emph{certified address}, which is used by the user to
transact with frozen assets. $\taxAuth$ maintains a mapping of taxpayers and
certified addresses, \ie for every taxpayer $\user$ it holds a list $A_\user$
of all certified taxation addresses that $\user$ requested.

A transaction $\tau = \langle \address_{s}, \address_{d}, \assets \rangle$,
which moves $\assets$ frozen assets from an address $\address_{s}$, is valid
only if $\address_{d} = \langle \address, \sig \rangle \land
\algoverify(\address, \sig, \verifykey_{\taxAuth}) = 1$. Consequently, miners
accept transactions that unfreeze assets only as long as said assets are
transferred to a certified address. Therefore, $\taxAuth$ can compute the
amount of $\user$'s assets as $\assets_\user := \sum_{i=1}^{n}
\balance(\address_{\user}[i])$, $n$ being the total number of $\user$'s
certified addresses.

We note that the system can accommodate multiple taxation authorities from
different countries. In that case, $\taxAuth$ is a federation of authorities,
each identified by a single key. Each authority's key is published on the
ledger and a taxpayer can certify their addresses and declare their assets to
the respective authorities.

Naturally, this mechanism introduces some challenges. Although standard
pay-to-public-key-hash addresses are $25$ bytes, certified addresses may be
significantly larger, due to the certification signature of $\taxAuth$. For
instance, ECDSA signatures in the DER format result in $72$ additional bytes,
thus making certified addresses $99$ bytes long. Nevertheless, certified
addresses are expected to be used only once, to declare the assets, thus the
overall storage cost should not be significant. Another important consideration
regards to the private state of the taxation authority; given the statute of
limitations, $\taxAuth$ might need to maintain its taxation private key and the
mapping of certified addresses for a significant period,
possibly resulting in significant maintenance costs.

We showcase our design via an auditable variation of Bitcoin ledger, denoted as
$\taxBtc$. $\taxBtc$ is initially parameterized by the public key of the
authority $(\signkey_{\taxAuth}, \verifykey_{\taxAuth})$, which is
part of the ledger's genesis block. During the execution, $\taxAuth$ can update
its key by simply signing a new key $\verifykey_{\taxAuth}'$ with
$\signkey_{\taxAuth}$ and publishing it on the ledger. A taxation period lasts
$52560$ blocks, \ie roughly $1$ calendar year, so block $52560$ and its
multiples are ``tax-auditing'' blocks.  When a tax-auditing block is issued,
all assets on $\taxBtc$ which are controlled by non-certified addresses are
frozen. To transact with assets from a frozen address, a user sends them to a
certified address, as described above.

Freezing complicates the system in a number of ways. First, the liveness
of a transaction~\cite{EC:GarKiaLeo15} may be affected. For
instance, a transaction which spends from a non-certified address will be
rejected, if it is created before but published after a tax-auditing block. We
sidestep this issue by enabling users to use certified addresses before the
freezing period, hence the liveness guarantees of the ledger apply
unconditionally on certified addresses. Second, it is possible that multiple
competing tax-auditing blocks are created, \eg multiple blocks which extend the
tax-auditing block. Therefore, $\taxAuth$ needs to pick one and certify it.
Afterwards, this certified block cannot be reverted and acts as a
``checkpoint''.

We note that $\taxBtc$ covers the desiderata proposed in
Section~\ref{sec:taxation}. Regarding privacy, although $\taxAuth$ can
de-anonymize the set of $\taxBtc$ users at a specific point in time, \ie when
the assets freeze, the users can employ standard Bitcoin addresses and
transactions outwith this period. Additionally, as with standard Bitcoin
addresses, third parties cannot obtain information regarding the identity of a
certified address's owner (as long as the signature itself does not leak it).
In terms of performance, a user can transact with their assets effortlessly, as
long as they use certified addresses to receive or unfreeze assets around the
taxation period. Importantly, users can certify their addresses ahead
of the freezing time, thus the additional load can be spread over a period
of a few days or weeks.
